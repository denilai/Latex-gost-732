% Также можно использовать \Referat, как в оригинале
\begin{abstract}
	Суть данной курсовой работы заключается в построении модели логического устройства сочетающего в себе функциональные узлы делителя частоты, фильтра дребезга контактов
	кнопки, конечного автомата генератора последовательности, матричного индикатора и генератора широтно-импульсных сигналов, а также верификация данной модели. 
	
	
	Работа состоит из пояснительной записки и графической части. Графическая часть представляет собой описание реализуемого программного решения.  
	
	Пояснительная записка содержит 5 разделов.
	 
	В первом разделе освещается проектирование моделей комбинационной логической схемы, которые будут использованы в дальнейших разделах.
	
	Второй раздел посвящен построению конечного автомата--генератора последовательности.
	
	В третьем разделе освещается процесс разработки функционального узла, применяющегося для анализа входной последовательности.
	
	Четвертый раздел отведен под описание процесса разработки узла для управления матричным индикатором.
	
	В пятом разделе описана реализация широтно-импульсного генератора и 64-разрядного сдвигового регистра.

	
	Работу выполнил студент группы ИВБО-02-19 Денисов К.Ю. 
	
	Руководитель --- профессор кафедры вычислительной техники, доцент технических наук Потехин~Д.~С.
	
	Консультант --- старший преподаватель кафедры вычислительной техники Борисенко~Н.~В.
	
	 Количество страниц --- \pageref*{LastPage}, количество рисунков --- \totfig, количество таблиц~---~\tottab, количество приложений --- 2. В работе использованы \totref~источников.
	 
	 Первое приложение содержит в себе исходные коды модулей на языке Verilog, разрабатываемых в ходе курсовой работы.
	 
	 Второе приложение содержит результаты симуляции цифрового устройства в САПР ISE Design Suite.
	 
	
\end{abstract}

%%% Local Variables: 
%%% mode: latex
%%% TeX-master: "rpz"
%%% End: 
