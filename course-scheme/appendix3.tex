\chapter{Руководство пользователя}
\label{cha:appendix3}
\setlength{\parindent}{12.5mm}

В данном приложении приведены правила использования цифрового устройства по ресинхронизации данных. 

Чтобы использовать аппаратный блок, спроектированный в ходе данной курсовой работы, в своих проектах, необходимо обратиться к списку выходов и входов, приведенному в разделе \ref{sec:design} настоящей курсовой работы и подключить модуль соответствующим образом.

Следует помнить, что данные на выход подаются по мере образования во входном буфере не менее четырех восьмиразрядных слов. Тактовые сигналы входного и выходного каналов независимы, однако тактовая частота для выхода устройства должна составлять как минимум  четверть от тактовой частоты входного интерфейса устройства.

Восьмиразрядные числа, переданные на вход, образуют тридцатидвухразрядную выходную последовательность в прямом порядке следования битов (big endian). 

Следует обратить внимание, что устройство осуществляет контроль записи и чтения --- сигнализирует о пустоте или заполненности внутренней памяти, но не блокирует запись или чтение самостоятельно. Управлять разрешением на запись и чтение должны внешние узлы. Схема лишь не допускает обращения к одному участку памяти для записи и чтения единовременно.

Устройство не поддерживает сброс внутренней памяти, так как обычно память в устройствах, реализуемых на ПЛИС представлена в виде блочной оперативной памяти и выделенной оперативной памяти. Эти ресурсы обычно не поддерживают сброс. Сброс возможно реализовать в регистрах. 


%\begin{lstlisting}[style=pseudocode,caption={Алгоритм оценки дипломных работ}]
%	sdfsadf sdf asdf sadf s
%\end{lstlisting}
%%% Local Variables: 
%%% mode: latex
%%% TeX-master: "rpz"
%%% End: 
