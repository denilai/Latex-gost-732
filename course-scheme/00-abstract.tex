% Также можно использовать \Referat, как в оригинале
\begin{abstract}
	Суть курсовой работы заключается в анализе проблематики разработки мультисинхронных цифровых устройств, разработке и реализации устройства для ресинхронизации данных, передаваемых из одного домена синхронизации цифрового устройства в другой
	
	
	
	Работа состоит из пояснительной записки и графической части. Графическая часть представляет описание программного продукта. 
	
	Пояснительная записка содержит 3 раздела
	

	
	Работу выполнил студент группы ИВБО-02-19 Денисов К.Ю. Руководитель --- Тарасов И.Е.
	
	 Работа содержит \pageref*{LastPage}~страницы, использованы \totref~источников, содержит \totfig~рисунков, \tottab~таблицу.
	

	
Это пример каркаса расчётно-пояснительной записки, желательный к использованию в РПЗ проекта по курсу РСОИ.

Данный опус, как и более новые версии этого документа, можно взять по адресу (\url{https://github.com/rominf/latex-g7-32}).

Текст в документе носит совершенно абстрактный характер.
\end{abstract}

%%% Local Variables: 
%%% mode: latex
%%% TeX-master: "rpz"
%%% End: 
