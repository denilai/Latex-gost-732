\Introduction



Целью данной курсовой работы является построение цифрового устройства ресинхронизации данных. Данное устройство должно реализовывать способ организации данных <<first in, first out>>, иметь восьмиразрядный вход и тридцатидвухразрядный выход.

Данные на выход должны подаваться по мере образования во входном буфере не менее четырех восьмиразрядных слов. Тактовые сигналы входного и выходного каналов независимы, однако тактовая частота для выхода устройства составляет как минимум  четверть от тактовой частоты входного интерфейса устройства.

Актуальность данной работы состоит в том, что большинство современных устройств, работающих с данными, включая микроконтроллеры, модемы, сетевые интерфейсы и сетевые процессоры, имеет множество последовательных интерфейсов и различных сигналов, которые производят передачу данных и сигналов управления через многие домены синхрочастоты.

Все разрабатываемые модули необходимо спроектировать с использованием языка описания аппаратуры Verilog.

Проектирование, разработку и тестирование модулей реализуемых устройств планируется осуществлять средствами ISIM в САПР ISE Design Suite.

Исходные данные к курсовой работе включают входные данные для записи в очередь FIFO.


%\begin{itemize}
%\item проанализировать предложенную ;
%\item спроектировать свою, новую всячину;
%\item изготовить всякую всячину;
%\item проверить её работоспособность.
%\end{itemize}
%
%Вот так-то. А этот абзац вставлен для визуальной оценки отступа от перечня до следующего абзаца.