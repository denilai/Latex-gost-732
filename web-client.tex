\chapter{Создание веб--клиента}

В данном разделе будет рассмотрен процесс реализации веб--приложения, с помощью которого будет производится взаимодействие конечного пользователя с созданной в рамках предыдущих работ базой данных. 

\section{Общие требования к приложению}

Приведем технические требования, предъявляемые к разрабатываемому приложению.

\subsection{Требования к функционалу}
К функционалу приложения предъявляются следующие требования:
\begin{itemize}
	\item регистрация и авторизация пользователя в системе;
	\item добавление, изменение, удаление, обновление информации по
	теме;
	\item фильтрация списков по соответствующим признакам;
	\item просмотр информации по запросу.
\end{itemize}

\subsection{Требования к интерфейсу}

Интерфейс системы должен поддерживать русский язык.

Интерфейс системы должен быть спроектирован с учетом ролевой
модели и уровней доступа пользователей.

Интерфейс системы должен обеспечивать наглядное, интуитивно
понятное представление структуры размещенной информации, быстрый и
логичный переход к соответствующим разделам.

Навигационные элементы интерфейса должны обеспечивать понимание
пользователем их смысла и обеспечивать навигацию по всем доступным
пользователю разделам и отображать соответствующую информацию.

Интерфейс системы должен позволять решать задачи пользователя
наиболее быстрым, простым и удобным из возможных способов.

Дизайн и удобство интерфейса должны быть на уровне ожиданий
современного пользователя и восприниматься им как комфортная, удобная и
приятная рабочая среда.

\section{Реализация приложения}

На основании функциональных требований, указанных в задании, было реализовано веб приложение, позволяющие взаимодействовать с данными, хранящимися в базе данных \texttt{cycle}, созданной в ходе выполнения предыдущих практических работ данного курса.

Интерфейс системы спроектирован с учетом ролевой
модели и уровней доступа пользователей и подразумевает регистрацию пользователя в СУБД MySQL.

Интерфейс поддерживает русский язык и позволяет пользователю удобно просматривать сведения, содержащиеся в таблицах выбранной базы данных.

Акцент в приложении сделан на содержании, дизайн интерфейса прост и лаконичен, чтобы облегчить и упростить взаимодействие пользователя с системой.

\subsection{Используемые технологии}
В качестве набора программных технологий был выбран LAMP--стэк, объединяющий в себе следующие компоненты:
\begin{itemize}
	\item Linux;
	\item Apache;
	\item MySQL;
	\item PHP.
\end{itemize}

Программно-аппаратная часть сервиса (бэкенд) реализована на языке PHP в парадигме объектно-ориентированного программирования, а также с использованием языка MySQL для работы с базой данных.

Структура гипертекстового документа описана средствами HTML и CSS.

Клиентская сторона пользовательского интерфейса системы (фронтэнд) основана на подходе AJAX (Asynchronous Javascript and XML). В качестве языка программирования был использован Java Script. 

Запросы веб-сервиса обрабатываются сервером Apache. 

\subsection{Возможные сценария использования веб-приложения}

Реализуемое веб--приложение может быть использовано для просмотра таблиц, добавления данных в таблицы базы данных под управлением СУБД MySQL, а также для написания произвольных DML и DDL запросов на диалекте MySQL. 

Веб--приложение объединяет в себе часть возможностей командой строки СУБД MySQL и веб-клиента для просмотра таблиц.


\subsection{Демонстрация работы веб-приложения}

Продемонстрируем общие сценарий использования веб--приложения, созданного в рамках данной практической работы. 

\subsubsection{Авторизация}
При первом входе на сайт пользователю необходимо авторизоваться и выбрать пользователя СУБД MySQL, который имеет права на изменение и просмотр данных в выбранной базе данных. Форма авторизации приведена на Рисунке \ref{fig:auth-empty}.
\begin{figure}[h!]
	\centering
	\includegraphics[width=1\linewidth]{figures/web-clent/auth-empty}
	\caption{Авторизация пользователя}
	\label{fig:auth-empty}
\end{figure}


 В данном примере будет выбран пользователь \texttt{denilai}, который имеет привилегии супер-пользователя в данной базе данных и может вносить изменения в базу даннных, просматривать все записи в ней, создавать новые постоянные и временные таблицы, процедуры и макросы.



\subsubsection{Отображение записей таблиц}
После успешной авторизации пользователю будет предложено выбрать таблицу в базе данных, записи в которой будут отображены на экране. Поля таблиц могут содержать кириллические символы --- они будут корректно отображаться веб-приложением (см. Рисунки \ref{fig:select-1}, \ref{fig:select-log}). %\cite{converse2004php5} \cite{baldinLatex} \cite{d2004mysql} \cite{yurt2011dev}

% TODO: \usepackage{graphicx} required
\begin{figure}[h!]
	\centering
	\includegraphics[width=1\linewidth]{web-clent/select-1}
	\caption{Отображение записей таблицы Component}
	\label{fig:select-1}
\end{figure}

% TODO: \usepackage{graphicx} required
\begin{figure}[h!]
	\centering
	\includegraphics[width=1\linewidth]{web-clent/select-1}
	\caption{Отображение записей таблицы Log}
	\label{fig:select-log}
\end{figure}



\subsubsection{Добавление записей в в таблицы}
Ниже, под выведенными записями таблицы расположены поля для добавления значений в выбранную базу данных. В случае ввода корректных данных, после нажатия кнопки <<OK>>, поля будут немедленно добавлены в таблицу и отобразятся на экране (см. Рисунок \ref{fig:insert-1}).

% TODO: \usepackage{graphicx} required
\begin{figure}[h!]
	\centering
	\includegraphics[width=1\linewidth]{web-clent/insert-1}
	\caption{Добавление записей в таблицу Frame}
	\label{fig:insert-1}
\end{figure}


\subsubsection{Выполнение произвольных запросов}
В самом низу веб--страницы расположено текстовое поле для выполнения SQL--запросов. Запросы могут относиться не только к выбранной таблице или базе данных, а к любой таблице, права на взаимодействие с которой имеет выбранный пользователь. 

Текст, введенный пользователем, передается в виде строки на исполнение MySQL серверу. В случае ввода корректного запроса, он будет выполнен, о чем будет сообщено пользователю. 

С помощью данного поля также можно производить операции по удалению и вставке записей в таблицы, отображаемые на экране (см. Рисунок \ref{fig:request}).

% TODO: \usepackage{graphicx} required
\begin{figure}[h!]
	\centering
	\includegraphics[width=1\linewidth]{web-clent/request}
	\caption{Выполнение произвольных запросов}
	\label{fig:request}
\end{figure}

С помощью данного веб--клиента пользователь может просматривать записи, хранящиеся в базе данных, отмечать продвижение деталей по производственным цехам, отражать текущий статус сборки велосипеда, проверять наличие тех или иных компонентов на складе, получать полную информацию о них. Интерфейс отвечает требованиям, выдвинутым в задании: поддерживает русский язык, акцент в приложении сделан на содержании, дизайн интерфейса прост и лаконичен, чтобы упросить взаимодействие пользователя с системой.


%Воспользуемся декартовым произведением таблиц 







%\begin{lstlisting}
%create table posts (
%	id_post int (10) AUTO_INCREMENT,
%	message text NOT NULL,
%	id_author int (10) NOT NULL,
%	id_topic int (10) NOT NULL,
%	PRIMARY KEY (id_post),
%	FOREIGN KEY (id_author) REFERENCES users (id_user),
%	FOREIGN KEY (id_topic) REFERENCES topics (id_topic)
%);
%\end{lstlisting}
