%% Преамбула TeX-файла

% 1. Стиль и язык
\documentclass[utf8x]{G7-32} % Стиль (по умолчанию будет 14pt)
\usepackage[T2A]{fontenc}
\usepackage[russian]{babel}
% Остальные стандартные настройки убраны в preamble.inc.tex.
%%%==========================================%%
%%% Для избежания переносов слов
%\usepackage{ragged2e}
%\usepackage{microtype}
%\usepackage{setspace}
%%\полуторный интервал
%\onehalfspacing
%\justifying
%\sloppy
%\tolerance=500
%\hyphenpenalty=10000
%\emergencystretch=3em
%%%==========================================%%

% Настройки стиля ГОСТ 7-32
% Для начала определяем, хотим мы или нет, чтобы рисунки и таблицы нумеровались в пределах раздела, или нам нужна сквозная нумерация.
\EqInChapter % формулы будут нумероваться в пределах раздела
\TableInChapter % таблицы будут нумероваться в пределах раздела
\PicInChapter % рисунки будут нумероваться в пределах раздела

% Добавляем гипертекстовое оглавление в PDF
\usepackage[
bookmarks=true, colorlinks=true, unicode=true,
urlcolor=black,linkcolor=black, anchorcolor=black,
citecolor=black, menucolor=black, filecolor=black,
]{hyperref}

%Величина «красной строки»
% Красная строка
\usepackage{indentfirst}
\setlength{\parindent}{1.25cm} % для шрифта 14pt

\makeatletter
\renewcommand{\@oddfoot}{\hfil \small \arabic{page}\hfil}
\renewcommand{\@evenfoot}{\hfil \small \arabic{page}\hfil}
\makeatother


\usepackage[tableposition=top,singlelinecheck=false]{caption}
%\usepackage{subcaption}


%подсчет сущностей

\usepackage{lastpage}

\usepackage{etoolbox}

\newcounter{totfigures}
\newcounter{tottables}
\newcounter{totreferences}
\newcounter{totappendix}
\pretocmd{\bibitem}{\addtocounter{totreferences}{1}}{}{}
\pretocmd{\chapter}{\addtocounter{totfigures}{\value{figure}}}{}{}
\pretocmd{\chapter}{\addtocounter{tottables}{\value{table}}}{}{}

\makeatletter
\AtEndDocument{%
	\addtocounter{totfigures}{\value{figure}}%
	\addtocounter{tottables}{\value{table}}%
	\immediate\write\@mainaux{%
		\string\gdef\string\totfig{\number\value{totfigures}}%
		\string\gdef\string\tottab{\number\value{tottables}}% 
		\string\gdef\string\totref{\number\value{totreferences}}%   
	}%
}
\makeatother



\DeclareCaptionLabelFormat{gostfigure}{Рисунок #2}
\DeclareCaptionLabelFormat{gosttable}{Таблица #2}
\DeclareCaptionLabelSeparator{gost}{~---~}
% Можно разбивать длинные таблицы вручную, оформляя каждую как table. В этом случае для продолжений таблицы нужно создать отдельный стиль заголовка
\DeclareCaptionLabelFormat{continued}{Продолжение таблицы~#2}
\captionsetup*[ContinuedFloat]{labelformat=continued}
\captionsetup{labelsep=gost}
\captionsetup*[figure]{labelformat=gostfigure, justification=centering,font={small, bf}}  % выравнивание по центру
\captionsetup*[table]{hangindent=-10pt, indention=0pt,parindent=0pt,margin=0pt,labelformat=gosttable,justification=raggedright,font={small,it}}
\captionsetup*[lstlisting]{font={small, it}}

% Изменение начертания шрифта --- после чего выглядит таймсоподобно.
% apt-get install scalable-cyrfonts-tex

\IfFileExists{cyrtimes.sty}
    {
        \usepackage{cyrtimespatched}
    }
    {
        % А если Times нету, то будет CM...
    }

\usepackage{graphicx}   % Пакет для включения рисунков
%расположение графики
\graphicspath{{images/}{figures/}{screnshots/}{../images/}}  % папки с картинками

% С такими оно полями оно работает по-умолчанию:
% \RequirePackage[left=20mm,right=10mm,top=20mm,bottom=20mm,headsep=0pt]{geometry}
% Если вас тошнит от поля в 10мм --- увеличивайте до 20-ти, ну и про переплёт не забывайте:
\geometry{right=10mm, left=30mm, top=20mm, bottom=20mm}
%\geometry{left=30mm}


% Пакет Tikz
\usepackage{tikz}
\usetikzlibrary{arrows,positioning,shadows}

% Произвольная нумерация списков.
\usepackage{enumerate}

% ячейки в несколько строчек
\usepackage{multirow}

% itemize внутри tabular
\usepackage{paralist,array}


% Настройки листингов.
\include{listings.inc}

% Полезные макросы листингов.
\include{macros.inc}

\renewcommand{\rmdefault}{ftm}

\begin{document}

\frontmatter % выключает нумерацию ВСЕГО; здесь начинаются ненумерованные главы: реферат, введение, глоссарий, сокращения и прочее.

% Команды \breakingbeforechapters и \nonbreakingbeforechapters
% управляют разрывом страницы перед главами.
% По-умолчанию страница разрывается.

% \nobreakingbeforechapters
% \breakingbeforechapters

%% Также можно использовать \Referat, как в оригинале
\begin{abstract}
	

	\newcounter{totfigures}
	\newcounter{tottables}
	\makeatletter
	\AtEndDocument{%
		\addtocounter{totfigures}{\value{figure}}%
		\addtocounter{tottables}{\value{table}}%
		\immediate\write\@mainaux{%
			\string\gdef\string\totfig{\number\value{totfigures}}%
			\string\gdef\string\tottab{\number\value{tottables}}%    
		}%
	}
	\makeatother
	
	
	\pretocmd{\chapter}{\addtocounter{totfigures}{\value{figure}}}{}{}
	\pretocmd{\chapter}{\addtocounter{tottables}{\value{table}}}{}{}
	% ...
	Дипломная работа: \pageref*{LastPage}~с., \totfig~рис., \tottab~табл...
	

	
Это пример каркаса расчётно-пояснительной записки, желательный к использованию в РПЗ проекта по курсу РСОИ.

Данный опус, как и более новые версии этого документа, можно взять по адресу (\url{https://github.com/rominf/latex-g7-32}).

Текст в документе носит совершенно абстрактный характер.
\end{abstract}

%%% Local Variables: 
%%% mode: latex
%%% TeX-master: "rpz"
%%% End: 


\tableofcontents

%\include{10-defines}
\Abbreviations %% Список обозначений и сокращений в тексте
\begin{description}
\item [AJAX] Asynchronous Javascript and XML
\item [CLI] Command line interface
\item [DDL] Data Definition Language
\item [DML] Data Manipulation Language
\item [CSS] Cascading Style Sheets
\item [HTML] HyperText Markup Language
\item[SQL] Structured query language.
\item[БД] База данных.
\item[ПО] Программное обеспечение.
\item[СУБД] Система управления базами данных.

\end{description}

%%% Local Variables:
%%% mode: latex
%%% TeX-master: "rpz"
%%% End:


\Introduction

Целью работы является построение логической и физической модели данных выбранной предметной области, освоение навыков работы с программным обеспечением ERwin Data Modeler, ErAssistant, MySQL Workbench, MySQL Server. 

На заключительном этапе работы необходимо представить веб--приложение или полнофункциональное приложение, организующее взаимодействие пользователя с созданной базой данных.  

%\begin{itemize}
%\item проанализировать предложенную ;
%\item спроектировать свою, новую всячину;
%\item изготовить всякую всячину;
%\item проверить её работоспособность.
%\end{itemize}
%
%Вот так-то. А этот абзац вставлен для визуальной оценки отступа от перечня до следующего абзаца.

\mainmatter % это включает нумерацию глав и секций в документе ниже

\newcommand{\erassistant}{ErAssistant~}

\chapter{Построение модели данных}
\label{cha:analysis}
%
% % В начале раздела  можно напомнить его цель
%
В данном разделе описывается процесс создания проекта базы данных в среде \erassistant \textit{(ver. 2.10).}

\section{Работа по методическим указаниям}
Произведем разработку проекта базы данных по теме, утвержденной в рамках дисциплины <<Разработка базы данных>>.
\subsection{Описание предметной области}
Компания занимается производством и продажей небольших статуэток,
раскрашиваемых вручную. Компания имеет несколько производственных
направлений. Миниатюры изготавливаются из гипса, олова или алюминия.

Компания распространяет свои товары по трем каналам. Компания содержит
пять собственных розничных магазинов. Помимо этого, компания владеет
сайтом, на котором ведется online-торговля, и осуществляет оптовые поставки
сторонним
дистрибьюторам.
Для
анализа
статистики,
системой автоматизации производства, нужен интерактивный аналитический
инструмент. Поэтому необходимо спроектировать и построить модель данных,
которая станет хранилищем информации по производству.

В ходе производства изделий система автоматизации производства
управляет всеми станками компании. Каждый станок реализует полный цикл
производства изделий, включая:
\begin{itemize}
	\item заполнение формы сырьем (гипсом, оловом или алюминием);
	\item затвердевание материала;
	\item  удаление изделия из формы после затвердевания;
	\item при необходимости автоматизированная раскраска изделий (оловянные фигурки не раскрашиваются);
	\item сушку после покраски (при необходимости).
\end{itemize}


Покраска и сушка могут производиться за несколько этапов в зависимости
от сложности изделия. По мере готовности изделия проходят проверку,
выполняемую оператором станка.

Оператор станка регистрируется в системе. В ходе этого процесса оператор
сообщает системе автоматизации производства тип производимых изделии и
объем загруженного в машину сырья. Оператор также делает в системе запись
при отбраковке изделий.

В ходе
интервью необходимые для эффективного анализа статистики:
\begin{itemize}
	
	\item  число принятых изделий по объему сырья, видам изделий, машинам и
	\item  время формовки и затвердевания по видам изделий, машинам и дням;
	\item  время покраски и сушки по типам краски, видам изделий, машинам и
	\item  сворачивание по подтипам изделий, которые сворачиваются по типам;
	\item  сворачивание по материалам (гипс, олово или алюминий);
	\item  сворачивание машин по фабрикам, которые сворачиваются по странам;
	\item  сворачивание дней по месяцам, месяцев — по кварталам;
	\item  возможность фильтрации информации по производителю и дате покупки машины.
\end{itemize}

Анализ файла-экспорта из системы автоматизации производства показал,
что для каждого вида производимых изделий есть отдельная строка, в которой
присутствует следующая информация:
\begin{itemize}
	\item  тип изделия;
	\item  объем сырья;
	\item  номер машины;
	\item  личный номер оператора;
	\item  время и дата начала производства (когда серия начата);
	\item  время и дата окончания производства (когда серия закончена);
	\item  флаг отбраковки.
\end{itemize}
%\newpage
\subsection{Построение модели}
По приведенному описанию предметной области построим ее модель в среде \erassistant. Укажем линии связей, назначим им имена, укажем типы и кратность связей. В результате работы, модель примет вид, приведенный на Рисунке \ref{fig:1-metod}.

% TODO: \usepackage{graphicx} required
\begin{figure}[ht]
	\centering
	\includegraphics[width=0.9\linewidth]{1-metod}
	\caption{Модель данных Производства со связями}
	\label{fig:1-metod}
\end{figure}
\section{Индивидуальное задание}
Произведем разработку проекта базы данных по индивидуальной теме с использованием программной среды ER Assistant
\subsection{Описание предметной области}
В качестве индивидуального задания была выбрана реализация модели информационной системы по хранению и анализу данных  предприятия, занимающегося сборкой и поставкой спортивных велосипедов для конечных потребителей по индивидуальному заказу.

Предприятие располагает широким выборов компонентов и комплектующих для сборки велосипедов следующих типов:
\begin{itemize}
	\item дорожный;
	\item горный;
	\item кросс-кантри;
	\item эндуро;
	\item прогулочный.
\end{itemize}

Предприятие может предложить сконфигурировать велосипед, отдельно выбрав каждый из предложенных компонентов:
\begin{itemize}
	\item рама;
	\item вилка;
	\item руль;
	\item трансмиссия;
	\item колеса;
	\item тормозная система.
\end{itemize}

Контроль над выполнением работ по сборке велосипеда проводится в виде учета всех операций по сборке, настройке и тестированию, проводимых на территории предприятия ее сотрудниками. При этом каждая запись содержит следующую информацию о проведенных работах:
\begin{itemize}
	\item внутренний номер изделия;
	\item время;
	\item этап работ;
	\item название цеха;
	\item имя мастера;
	\item статус;
	\item примечание.
\end{itemize}

На предприятии ведется учет всех компонентов велосипедов.
В базе данных предприятия хранится информация о каждом компоненте, приобретенном у партнеров или изготовленном самостоятельно.

В независимости от типа компонента он обладает общей информацией о наименовании производителя, месте и времени изготовления, типе и рекомендованной розничной цене. 
Также каждый компонент имеет особые сведения, присущие данному типу детали.

\subsection{Построение модели}
После приведения общих сведений о роде деятельности предприятия, факторизируем модель данных информационной системы предприятия в среде \erassistant (cм Рисунок \ref{fig:1-cycle}). Приложение \erassistant позволяет пользователю создавать, редактировать диаграммы сущностей и связей.

Укажем названия связей, их идентификаторы и кратность, исходя из вида отношений, выстроенных между сущностями.
% TODO: \usepackage{graphicx} required
\begin{figure}[h!]
	\centering
	\includegraphics[width=1\linewidth]{1-cycle}
	\caption{Модель данных вело-предприятия}
	\label{fig:1-cycle}
\end{figure}



На данном этапе выполнения работы мы реализовали проект базы данных будущего хранилища данных, планируемого к применению в компании, занимающейся сборкой и поставкой велосипедов по индивидуальному заказу. Был проведен анализ переметной области, определен список сущностей, которые наиболее полно смогут описать сущности, участвующие в производственном процессе.
%%% Local Variables:
%%% mode: latex
%%% TeX-master: "rpz"
%%% End:

\clearpage
\newcommand{\erdatamodaler}{ERwin Data Modeler~}

\chapter{Создание логической и физической модели данных}
\label{cha:dmd}
В данном разделе будет рассмотрено создание логической и физической модели данных предложенных предметных областей в ПО ERwin Data Modeler.

На этапе инфологического проектирования базы данных должна быть построена
модель предметной области, не привязанная к конкретной СУБД, понятная не только
разработчикам информационной системы, но и экономистам, менеджерам и другим
специалистам. В то же время модель предметной области должна максимально точно
отражать семантику предметной области и позволять легко перейти к модели данных
конкретной СУБД.

\textbf{Логический уровень} --- это уровень, соответствующий инфологическому этапу проектирования
и не привязанный к конкретной СУБД. Модели логического уровня оперируют с
понятиями сущностей, атрибутов и связей, которые на этом уровне именуются на
естественном языке (в нашем случае – на русском) так, как они называются в
реальном мире.

\textbf{Физический уровень} --- это отображение логической модели на модель данных
конкретной СУБД. Одной логической модели может соответствовать несколько
физических моделей. Причем, Erwin (как и другие CASE-системы проектирования баз
данных) позволяет автоматизировать отображение логической модели на физическую.

\section{Работа по методическим указаниям}

Порядок построения модели данных в среде \erdatamodaler рассмотрим на примере
автоматизированной информационной системы <<Реализация средств вычислительной
техники>>, предназначенной для учета продаж настольных компьютеров по заказам
клиентов.

Создание модели данных начинается с разработки логической модели, которая
должна представлять состав сущностей предметной области с перечнем атрибутов и
отношений между ними.

Результат разработки логической модели данных системы <<Реализация средств
вычислительной техники>>, предназначенной для учета продаж настольных
компьютеров по заказам клиентов приведен на Рисунке \ref{fig:2-logical-model-method}.

% TODO: \usepackage{graphicx} required
\begin{figure}[htpb]
	\centering
	\includegraphics[angle=0,width=\linewidth]{/2-method}
	\caption{Логическая модель данных системы <<Реализация средств вычислительной техники>>}
	\label{fig:2-logical-model-method}
\end{figure}

Для построения физической модели данных системы, следует определиться с СУБД, в которой будет реализована модель. При построении физической модели данных следует учитывать формальную теория представления и обработки данных в конкретной системе управления базами данных (СУБД).

В данной практической работе в качестве СУБД выбрана MySQL.

Приступим к построению физической модели данных системы <<Реализация средств вычислительной техники>>. Результат работы можно видеть на Рисунке \ref{fig:2-phisical-model}.
%\newpage
% TODO: \usepackage{graphicx} required
\begin{figure}[ht]
	\centering
	\includegraphics[angle=0,width=\linewidth]{2-phisical-model}
	\caption{Физическая модель данных системы <<Реализация средств вычислительной техники>>}
	\label{fig:2-phisical-model}
\end{figure}
\newpage
\subsection{Индивидуальное задание}
Приступим к построению логической модели данных системы <<Велосипедное предприятие>>. 
В соответствии с моделью, реализованной в ходе первой практической работы, добавим в рабочую область следующие сущности:
\begin{itemize}
	\item Component;
	\item FrameInfo;
	\item Frame;
	\item Frameset;
	\item FrameSize;
	\item Fork;
	\item ComponentType;
	\item Wheelset;
	\item Groupset;
	\item Brake;
	\item FctCycleBuild;
	\item CycleType;
	\item Bar;
	\item Setup.
\end{itemize}
Добавим связи между сущностями в соответствии с ранее построенной моделью. Логическая модель системы <<Велосипедное предприятие>> приведена на Рисунке \ref{fig:2-cycle-logical}.

% TODO: \usepackage{graphicx} required
\begin{figure}[h!]
	\centering
	\includegraphics[width=0.8\linewidth]{2-cycle-logical}
	\caption{Логическая модель данных системы <<Велосипедное предприятие>>}
	\label{fig:2-cycle-logical}
\end{figure}

После уточнения типов данных, выбранных в соответствии с предметной областью и спецификой СУБД MySQL.
Физическая модель системы <<Велосипедное предприятие>> приведена на Рисунке \ref{fig:2-cycle-phisical}.

% TODO: \usepackage{graphicx} required
\begin{figure}[h!]
	\centering
	\includegraphics[width=0.8\linewidth]{2-cycle-phisical}
	\caption{Физическая модель данных системы <<Велосипедное предприятие>>}
	\label{fig:2-cycle-phisical}
\end{figure}

После реализации физической и логической модели можно приступать к реализации модели данной системы в СУБД MySQL.
\hfill
\clearpage
\chapter{Создание базы данных}
\section{Работа по методическим указаниям}
Создадим базу данных \texttt{forum}, которая хранит в себе сведения о пользователях форумах и размещенных ими темах.

Помимо суперпользователя root, был создан пользователь denilai, под которым производятся все манипуляции с данными.

Создадим базу данных \texttt{forum} с помощью команды \texttt{CREATE DATABASE forum;}


Создадим таблицу \texttt{users} (см. Рисунок \ref{fig:create-users}):

% TODO: \usepackage{graphicx} required
\begin{figure}[h!]
	\centering
	\includegraphics[width=0.7\linewidth]{create-users}
	\caption{Создание базы данных users}
	\label{fig:create-users}
\end{figure}

%\begin{lstlisting}
%create table users (
%	id_user INT (10) AUTO_INCREMENT,
%	name varchar(20) NOT NULL,
%	email varchar(50) NOT NULL,
%	password varchar(15) NOT NULL,
%	PRIMARY KEY (id_user)
%);
%\end{lstlisting}

Создадим таблицу \texttt{topics} (см. Рисунок \ref{fig:create-topics}):

% TODO: \usepackage{graphicx} required
\begin{figure}[h!]
	\centering
	\includegraphics[width=0.7\linewidth]{create-topics}
	\caption{Создание базы данных topics}
	\label{fig:create-topics}
\end{figure}

%\begin{lstlisting}
%create table topics (
%	id_topic INT (10) AUTO_INCREMENT,
%	topic_name varchar(100) NOT NULL,
%	id_author INT (10) NOT NULL,
%	PRIMARY KEY (id_topic),
%	FOREIGN KEY (id_author) REFERENCES users (id_user)
%);
%\end{lstlisting}
Создадим таблицу \texttt{posts} (см. Рисунок \ref{fig:create-posts}):

\begin{figure}[h!]
	\centering
	\includegraphics[width=0.7\linewidth]{create-posts}
	\caption{Создание базы данных posts}
	\label{fig:create-posts}
\end{figure}

После создания таблиц, заполним их данными о пользователях форума, о темах и размещенных публикациях. Выполним операцию выборки данных без условия, чтобы увидеть все записи, занесенные в таблицы с помощью команды \texttt{SELECT * FROM <table-name>} (см. Рисунок \ref{fig:select-from-all}). 

% TODO: \usepackage{graphicx} required
\begin{figure}[h!]
	\centering
	\includegraphics[width=0.8\linewidth]{select-from-all}
	\caption{Операция выборки из всех таблиц }
	\label{fig:select-from-all}
\end{figure}

Выполним запрос \texttt{SELECT mesage, topic\_name FROM posts p JOIN topics t ON t.id\_author = p.id\_author; } для объединения данных из таблиц \texttt{topics} и \texttt{posts} по ключу id\_author и получения полной информации о сообщении и названию темы, в которой оно было размещено (см. Рисунок \ref{fig:join-post-author}).

% TODO: \usepackage{graphicx} required
\begin{figure}[ht]
	\centering
	\includegraphics[width=0.8\linewidth]{join-post-author}
	\caption{Запрос объединения}
	\label{fig:join-post-author}
\end{figure}

Выполним запрос выборки данных, явно указав поля отношения. Для этого перечислим имена полей через запятую после зарезервированного слова \texttt{SELECT} (см. Рисунок \ref{fig:field-select}).

% TODO: \usepackage{graphicx} required
\begin{figure}[h!]
	\centering
	\includegraphics[width=0.5\linewidth]{field-select}
	\caption{Операция выборки с указанием полей}
	\label{fig:field-select}
\end{figure}


Выполним более сложные запросы выборки, отсортировав записи в таблице \texttt{topics} по убыванию значения поля \texttt{topic\_name} и \texttt{id\_author}, а также опишем условие сравнения значения поля \texttt{id\_author} в сецкции \texttt{WHERE}(см. Рисунок \ref{fig:order-where-forum}). 

% TODO: \usepackage{graphicx} required
\begin{figure}[h!]
	\centering
	\includegraphics[width=0.5\linewidth]{order-where-forum}
	\caption{Сложные запросы выборки с сортировкой и условием}
	\label{fig:order-where-forum}
\end{figure}

Выполним операции по модификации таблицы --- добавим в таблицу \texttt{users} поле \texttt{country} типа \texttt{varchar (20)} со значением по умолчанию "Russia", а также добавим в эту же таблицу поле \texttt{age int(10)} со значением по умолчанию 19. 

Выведем все записи из таблицы \texttt{users}, обнаружим, что столбцы были вставлены успешно (см. Рисунок \ref{fig:alter-add-forum}).

% TODO: \usepackage{graphicx} required
\begin{figure}[h!]
	\centering
	\includegraphics[width=0.9\linewidth]{alter-add-forum}
	\caption{Добавление полей в таблицу \texttt{users}}
	\label{fig:alter-add-forum}
\end{figure}

В ходе данной практической работы были рассмотрены операторы DDL и DML диалекта MySQL. С помощью данных операторов была создана база учебная база данных \texttt{forum}, содержащая о пользователях форумах и размещенных ими темах.

\section{Индивидуальное задание}

Продолжим работу над созданием модели данных велосипедного предприятия. Создадим базу данных \texttt{cycle}, сущностями  которой будут таблицы, описания которых были проработаны в прошлых практических работах. 

Описание проектируемых отношений базы данных \texttt{cycle} приведены в таблицах \ref{tab:bar}--\ref{tab:log}.

%(\textbf{Bar})
\begin{table}[h!] 
	\centering
	\caption{Описание таблицы Bar}
	\begin{tabular}{|l|l|}
		\hline \textbf{Имя} & \textbf{Тип} \\
		\hline
		Width & INTEGER NOT NULL \\ \hline
		Diameter & INTEGER NOT NULL \\ \hline
		BarID & INTEGER NOT NULL AUTO\_INCREMENT \\ \hline
		Color & VARCHAR(20) NULL DEFAULT 'black' \\ \hline
	\end{tabular}
	\label{tab:bar}
\end{table}

%(\textbf{Brake})
\begin{table}[h!] 
	\centering
	\caption{Описание таблицы Brake}
	\begin{tabular}{|l|l|}
		\hline \textbf{Имя} & \textbf{Тип} \\
		\hline
		BrakeID &       INT NOT NULL AUTO\_INCREMENT \\ \hline
		ComponentID &   INTEGER NULL \\ \hline
	\end{tabular}
	\label{tab:brake}
\end{table}


%(\textbf{Component})

\begin{table}[h!] 
	\centering
	\caption{Описание таблицы Component}
	\begin{tabular}{|l|l|}
		\hline \textbf{Имя} & \textbf{Тип} \\
		\hline
		ComponentID &   INTEGER NOT NULL AUTO\_INCREMENT \\ \hline
		Brand & VARCHAR(20) NOT NULL \\ \hline
		ManufacturerCountry &  VARCHAR(20) NOT NULL \\ \hline
		PriceCent & INTEGER NOT NULL \\ \hline
		Model & VARCHAR(20) NOT NULL \\ \hline
		ComponentTypeID &  INTEGER NOT NULL \\ \hline
		WeightGramm &   INTEGER NOT NULL \\ \hline
	\end{tabular}
	\label{tab:component}
\end{table}

%(\textbf{ComponentType}) 
\begin{table}[h!] 
	\centering
	\caption{Описание таблицы ComponentType}
	\begin{tabular}{|l|l|}
		\hline \textbf{Имя} & \textbf{Тип} \\
		\hline
		ComponentTypeID &  INTEGER NOT NULL AUTO\_INCREMENT \\ \hline
		Name & VARCHAR(20) \\ \hline
	\end{tabular}
	\label{tab:componenttype}
\end{table}


%(\textbf{CycleType})

\begin{table}[h!] 
	\centering
	\caption{Описание таблицы CycleType}
	\begin{tabular}{|l|l|}
		\hline \textbf{Имя} & \textbf{Тип} \\
		\hline
		CycleTypeID &   INTEGER NOT NULL AUTO\_INCREMENT \\ \hline
		Name &  VARCHAR(20) NOT NULL \\ \hline
	\end{tabular}
	\label{tab:cycletype}
\end{table}

\begin{table}[h!] 
	\centering
	\caption{Описание таблицы FctCycleBuild}
	\begin{tabular}{|l|l|}
		\hline \textbf{Имя} & \textbf{Тип} \\
		\hline
		SetupID &       INTEGER NOT NULL \\ \hline
		Datetime &      DATE NOT NULL \\ \hline
		Stage & INTEGER NOT NULL \\ \hline
		Workshop &      varchar(20) NOT NULL DEFAULT 'main' \\ \hline
		Master &        VARCHAR(20) NOT NULL \\ \hline
		Status &        VARCHAR(20) NOT NULL \\ \hline
		Comment &       VARCHAR(20) NOT NULL \\ \hline
	\end{tabular}
	\label{}
\end{table}

%(\textbf{Fork})

\begin{table}[h!] 
	\centering
	\caption{Описание таблицы Fork}
	\begin{tabular}{|l|l|}
		\hline \textbf{Имя} & \textbf{Тип} \\
		\hline
		ForkID &  INTEGER NOT NULL AUTO\_INCREMENT \\ \hline
		Color & VARCHAR(20) NULL DEFAULT 'black' \\ \hline
		ComponentID &  INTEGER NULL \\ \hline
	\end{tabular}
	\label{ta:fork}
\end{table}


%(\textbf{Frame})

\begin{table}[h!] 
	\centering
	\caption{Описание таблицы Frame}
	\begin{tabular}{|l|l|}
		\hline \textbf{Имя} & \textbf{Тип} \\
		\hline
		FrameID &  INTEGER NOT NULL AUTO\_INCREMENT \\ \hline
		Color & VARCHAR(20) NULL DEFAULT 'black' \\ \hline
	\end{tabular}
	\label{tab:frame}
\end{table}


%(\textbf{FrameInfo})

\begin{table}[h!] 
	\centering
	\caption{Описание таблицы FrameInfo}
	\begin{tabular}{|l|l|}
		\hline \textbf{Имя} & \textbf{Тип} \\
		\hline
		FrameID &       INTEGER NOT NULL \\ \hline
		Size & INTEGER NOT NULL \\ \hline
		Stack & INTEGER NOT NULL \\ \hline
		Reach & INTEGER NOT NULL \\ \hline
	\end{tabular}
	\label{tab:frameinfo}
\end{table}


%(\textbf{Frameset})

\begin{table}[h!] 
	\centering
	\caption{Описание таблицы Frameset}
	\begin{tabular}{|l|l|}
		\hline \textbf{Имя} & \textbf{Тип} \\
		\hline
		FrameSizeID &   INTEGER NOT NULL AUTO\_INCREMENT \\ \hline
		ForkID &        INTEGER NOT NULL \\ \hline
		FramesetID &    INTEGER NOT NULL \\ \hline
		ComponentID &   INTEGER NOT NULL \\ \hline
	\end{tabular}
	\label{tab:frameset}
\end{table}


%(\textbf{FrameSize})

\begin{table}[h!] 
	\centering
	\caption{Описание таблицы FrameSize}
	\begin{tabular}{|l|l|}
		\hline \textbf{Имя} & \textbf{Тип} \\
		\hline
		FrameID &       INTEGER NOT NULL \\ \hline
		Size &  INTEGER NOT NULL \\ \hline
		FrameSizeID &   INTEGER NOT NULL AUTO\_INCREMENT \\ \hline
		ComponentID &   INTEGER NOT NULL \\ \hline
	\end{tabular}
	\label{tab:framesize}
\end{table}


%(%(\textbf{Groupset}))

\begin{table}[h!] 
	\centering
	\caption{Описание таблицы Groupset}
	\begin{tabular}{|l|l|}
		\hline \textbf{Имя} & \textbf{Тип} \\
		\hline
		Ratio & INTEGER NOT NULL \\ \hline
		BrakeID &       CHAR(18) NOT NULL \\ \hline
		GroupsetID &    INTEGER NOT NULL AUTO\_INCREMENT \\ \hline
		Color & VARCHAR(20) NOT NULL \\ \hline
		ComponentID &   INTEGER NOT NULL \\ \hline
	\end{tabular}
	\label{tab:groupset}
\end{table}


%(\textbf{Setup})

\begin{table}[h!] 
	\centering
	\caption{Описание таблицы Setup}
	\begin{tabular}{|l|l|}
		\hline \textbf{Имя} & \textbf{Тип} \\
		\hline
		GroupsetID &    INTEGER NOT NULL \\ \hline
		WheelsetID &    INTEGER NOT NULL \\ \hline
		FramesetID &    INTEGER NOT NULL \\ \hline
		BarID & INTEGER NOT NULL \\ \hline
		CycleTypeID &   INTEGER NOT NULL \\ \hline
		SetupID &       INTEGER NOT NULL AUTO\_INCREMENT \\ \hline
	\end{tabular}
	\label{tab:setup}
\end{table}


%(\textbf{Wheelset})

\begin{table}[h!] 
	\centering
	\caption{Описание таблицы Wheelset}
	\begin{tabular}{|l|l|}
		\hline \textbf{Имя} & \textbf{Тип} \\
		\hline
		DiametrInch &   INTEGER NOT NULL \\ \hline
		Width & INTEGER NOT NULL \\ \hline
		SpokesCount &   INTEGER NULL \\ \hline
		WheelsetID &    INTEGER NOT NULL AUTO\_INCREMENT \\ \hline
		Color & VARCHAR(20) NOT NULL \\ \hline
		ComponentID &   INTEGER NOT NULL \\ \hline
	\end{tabular}
	\label{tab:wheelset}
\end{table}

\begin{table}[h!] 
	\centering
	\caption{Описание таблицы Log}
	\begin{tabular}{|l|l|}
		\hline \textbf{Имя} & \textbf{Тип} \\
		\hline
		ID &   INTEGER NOT NULL AUTO\_INCREMENT \\ \hline
		msg & VARCHAR{100} NOT NULL \\ \hline
		row\_id & INTEGER NOT NULL \\ \hline
	\end{tabular}
	\label{tab:log}
\end{table}



\newpage\hfill\newpage\hfill\newpage

\subsection{Использование MySQL CLI Client}

Создадим данные таблицы с помощью MySQL CLI Client~---~клиента, предоставляющего доступ к СУБД через интерфейс командой строки, использовав ключевое слово \texttt{CREATE TABLE}. После создания таблиц выполним команду \texttt{SHOW TABLES}, выбрав базу данных \texttt{cycle} (см. Рисунок \ref{fig:show-tables-cycle}).

% TODO: \usepackage{graphicx} required
\begin{figure}[h!]
	\centering
	\includegraphics[width=0.5\linewidth]{show-tables-cycle}
	\caption{таблицы базы данных \texttt{cycle}}
	\label{fig:show-tables-cycle}
\end{figure}
Выведем записи из таблицы \texttt{Component} (см. Рисунок \ref{fig:select-components}).

% TODO: \usepackage{graphicx} required
\begin{figure}[h!]
	\centering
	\includegraphics[width=\linewidth]{select-components}
	\caption{Вывод записей из таблицы Components}
	\label{fig:select-components}
\end{figure}

Выполним более сложные запросы выборки --- объединим таблицы с помощью оператора \texttt{INNER JOIN} \texttt{Component} и \texttt{fork} по полю \linebreak \texttt{component\_id}, а также воспользуемся функцией \texttt{ROW\_NUMBER()} в сочетании с оконной функцией \texttt{OVER()}, пронумеровав компоненты из таблицы \texttt{Fork} одного цвета по возрастанию значения поля \texttt{color} (см. Рисунок \ref{fig:join-window-cycle}).

% TODO: \usepackage{graphicx} required
\begin{figure}[h!]
	\centering
	\includegraphics[width=\linewidth]{join-window-cycle}
	\caption{Сложные запросы выборки. База данных cycle}
	\label{fig:join-window-cycle}
\end{figure}


Создадим триггеры \texttt{delete\_component} и  \texttt{add\_component} на добавление и удаление записи в таблице \texttt{Component}. При срабатываении данного триггера, будет добавляться запись в таблицу log, информирующая о совершении манипуляций с данными (см. Листинг \ref{list:trigger}).
%\newpage
Приведем объявление данного триггера на диалекте MySQL:
\lstset{
	language=sql
}
\begin{lstlisting}[caption=Триггер, label={list:trigger}]
DELIMITER $$
drop trigger delete_component; $$

create trigger `delete_component` after delete on component
for each row begin
insert into log (msg, row_id) values (concat('delete component ',old.Brand,' ', old.Model), old.ComponentID);
end; $$

create trigger `add_component` after insert on component
for each row begin
insert into log (msg, row_id) values (concat('insert component ',new.Brand,' ', new.Model), new.ComponentID);
end; $$
DELIMITER $$
\end{lstlisting}

\subsection{Использование MySQL Workbench}

Выполним операции по изменению и просмотру данных, занесенных в базу данных \texttt{cycle}, с помощью инструмента для визуального проектирования баз данных MySQL Workbench.

Для этого подключимся к локально развернутому на машине MySQL Server, указав порт, имя пользователя и пароль. После этого мы получим доступ к пользовательскому интерфейсу программы, представляющему собой две области~---~область выполнения запросов и область отображения результатов (см. Рисунок \ref{fig:workbrench-interface}).

\begin{figure}[h!]
	\centering
	\includegraphics[width=\linewidth]{figures/web-clent/workbrench-interface}
	\caption{Интерфейс MySQL Workbench}
	\label{fig:workbrench-interface}
\end{figure}

Создадим временную таблицу, в котором объединим сведения о компонентах (см. Рисунок \ref{fig:workbrench-components}).
\begin{figure}[h!]
	\centering
	\includegraphics[width=1\linewidth]{figures/web-clent/workbrench-components}
	\caption{Просмотр временной таблицы компонентов}
	\label{fig:workbrench-components}
\end{figure}

Выполним еще один запрос~---~ограничим вывод только компонентами типа <<Fork>> (см. Рисунок \ref{fig:workbrench-forks}).

\begin{figure}[h!]
	\centering
	\includegraphics[width=1\linewidth]{figures/web-clent/workbrench-forks}
	\caption{Просмотр временной таблицы компонентов. Компоненты <<Fork>>}
	\label{fig:workbrench-forks}
\end{figure}

Данный визуальный инструмент позволяет удобно организовать работу с базой данных, сохранять SQL--скрипты, параметры подключения и настройки.
\clearpage
\chapter{Создание веб--клиента}

В данном разделе будет рассмотрен процесс реализации веб--приложения, с помощью которого будет производится взаимодействие конечного пользователя с созданной в рамках предыдущих работ базой данных. 

\section{Общие требования к приложению}

Приведем технические требования, предъявляемые к разрабатываемому приложению.

\subsection{Требования к функционалу}
К функционалу приложения предъявляются следующие требования:
\begin{itemize}
	\item регистрация и авторизация пользователя в системе;
	\item добавление, изменение, удаление, обновление информации по
	теме;
	\item фильтрация списков по соответствующим признакам;
	\item просмотр информации по запросу.
\end{itemize}

\subsection{Требования к интерфейсу}

Интерфейс системы должен поддерживать русский язык.

Интерфейс системы должен быть спроектирован с учетом ролевой
модели и уровней доступа пользователей.

Интерфейс системы должен обеспечивать наглядное, интуитивно
понятное представление структуры размещенной информации, быстрый и
логичный переход к соответствующим разделам.

Навигационные элементы интерфейса должны обеспечивать понимание
пользователем их смысла и обеспечивать навигацию по всем доступным
пользователю разделам и отображать соответствующую информацию.

Интерфейс системы должен позволять решать задачи пользователя
наиболее быстрым, простым и удобным из возможных способов.

Дизайн и удобство интерфейса должны быть на уровне ожиданий
современного пользователя и восприниматься им как комфортная, удобная и
приятная рабочая среда.

\section{Реализация приложения}

На основании функциональных требований, указанных в задании, было реализовано веб приложение, позволяющие взаимодействовать с данными, хранящимися в базе данных \texttt{cycle}, созданной в ходе выполнения предыдущих практических работ данного курса.

Интерфейс системы спроектирован с учетом ролевой
модели и уровней доступа пользователей и подразумевает регистрацию пользователя в СУБД MySQL.

Интерфейс поддерживает русский язык и позволяет пользователю удобно просматривать сведения, содержащиеся в таблицах выбранной базы данных.

Акцент в приложении сделан на содержании, дизайн интерфейса прост и лаконичен, чтобы облегчить и упростить взаимодействие пользователя с системой.

\subsection{Используемые технологии}
В качестве набора программных технологий был выбран LAMP--стэк, объединяющий в себе следующие компоненты:
\begin{itemize}
	\item Linux;
	\item Apache;
	\item MySQL;
	\item PHP.
\end{itemize}

Программно-аппаратная часть сервиса (бэкенд) реализована на языке PHP в парадигме объектно-ориентированного программирования, а также с использованием языка MySQL для работы с базой данных.

Структура гипертекстового документа описана средствами HTML и CSS.

Клиентская сторона пользовательского интерфейса системы (фронтэнд) основана на подходе AJAX (Asynchronous Javascript and XML). В качестве языка программирования был использован Java Script. 

Запросы веб-сервиса обрабатываются сервером Apache. 

\subsection{Возможные сценария использования веб-приложения}

Реализуемое веб--приложение может быть использовано для просмотра таблиц, добавления данных в таблицы базы данных под управлением СУБД MySQL, а также для написания произвольных DML и DDL запросов на диалекте MySQL. 

Веб--приложение объединяет в себе часть возможностей командой строки СУБД MySQL и веб-клиента для просмотра таблиц.


\subsection{Демонстрация работы веб-приложения}

Продемонстрируем общие сценарий использования веб--приложения, созданного в рамках данной практической работы. 

\subsubsection{Авторизация}
При первом входе на сайт пользователю необходимо авторизоваться и выбрать пользователя СУБД MySQL, который имеет права на изменение и просмотр данных в выбранной базе данных. Форма авторизации приведена на Рисунке \ref{fig:auth-empty}.
\begin{figure}[h!]
	\centering
	\includegraphics[width=1\linewidth]{figures/web-clent/auth-empty}
	\caption{Авторизация пользователя}
	\label{fig:auth-empty}
\end{figure}


 В данном примере будет выбран пользователь \texttt{denilai}, который имеет привилегии супер-пользователя в данной базе данных и может вносить изменения в базу даннных, просматривать все записи в ней, создавать новые постоянные и временные таблицы, процедуры и макросы.



\subsubsection{Отображение записей таблиц}
После успешной авторизации пользователю будет предложено выбрать таблицу в базе данных, записи в которой будут отображены на экране. Поля таблиц могут содержать кириллические символы --- они будут корректно отображаться веб-приложением (см. Рисунки \ref{fig:select-1}, \ref{fig:select-log}). %\cite{converse2004php5} \cite{baldinLatex} \cite{d2004mysql} \cite{yurt2011dev}

% TODO: \usepackage{graphicx} required
\begin{figure}[h!]
	\centering
	\includegraphics[width=1\linewidth]{web-clent/select-1}
	\caption{Отображение записей таблицы Component}
	\label{fig:select-1}
\end{figure}

% TODO: \usepackage{graphicx} required
\begin{figure}[h!]
	\centering
	\includegraphics[width=1\linewidth]{web-clent/select-1}
	\caption{Отображение записей таблицы Log}
	\label{fig:select-log}
\end{figure}



\subsubsection{Добавление записей в в таблицы}
Ниже, под выведенными записями таблицы расположены поля для добавления значений в выбранную базу данных. В случае ввода корректных данных, после нажатия кнопки <<OK>>, поля будут немедленно добавлены в таблицу и отобразятся на экране (см. Рисунок \ref{fig:insert-1}).

% TODO: \usepackage{graphicx} required
\begin{figure}[h!]
	\centering
	\includegraphics[width=1\linewidth]{web-clent/insert-1}
	\caption{Добавление записей в таблицу Frame}
	\label{fig:insert-1}
\end{figure}


\subsubsection{Выполнение произвольных запросов}
В самом низу веб--страницы расположено текстовое поле для выполнения SQL--запросов. Запросы могут относиться не только к выбранной таблице или базе данных, а к любой таблице, права на взаимодействие с которой имеет выбранный пользователь. 

Текст, введенный пользователем, передается в виде строки на исполнение MySQL серверу. В случае ввода корректного запроса, он будет выполнен, о чем будет сообщено пользователю. 

С помощью данного поля также можно производить операции по удалению и вставке записей в таблицы, отображаемые на экране (см. Рисунок \ref{fig:request}).

% TODO: \usepackage{graphicx} required
\begin{figure}[h!]
	\centering
	\includegraphics[width=1\linewidth]{web-clent/request}
	\caption{Выполнение произвольных запросов}
	\label{fig:request}
\end{figure}

С помощью данного веб--клиента пользователь может просматривать записи, хранящиеся в базе данных, отмечать продвижение деталей по производственным цехам, отражать текущий статус сборки велосипеда, проверять наличие тех или иных компонентов на складе, получать полную информацию о них. Интерфейс отвечает требованиям, выдвинутым в задании: поддерживает русский язык, акцент в приложении сделан на содержании, дизайн интерфейса прост и лаконичен, чтобы упросить взаимодействие пользователя с системой.


%Воспользуемся декартовым произведением таблиц 







%\begin{lstlisting}
%create table posts (
%	id_post int (10) AUTO_INCREMENT,
%	message text NOT NULL,
%	id_author int (10) NOT NULL,
%	id_topic int (10) NOT NULL,
%	PRIMARY KEY (id_post),
%	FOREIGN KEY (id_author) REFERENCES users (id_user),
%	FOREIGN KEY (id_topic) REFERENCES topics (id_topic)
%);
%\end{lstlisting}

%\cite{d2004mysql},\cite{baldinLatex}, \cite{converse2004php5}, \cite{yurt2011dev}.

\backmatter %% Здесь заканчивается нумерованная часть документа и начинаются ссылки и
            %% заключение

\Conclusion % заключение к отчёту

В ходе выполнения данных практических работ были получены знания по написанию запросов на диалекте MySQL, отработаны навыки работы в средах ER Assistant и ERwin Data Modeler, навыки работы с интерфейсом командной строки MySQL CLI Client, а также с инструментом для визуального проектирования баз данных MySQL Workbench. 

Полученные знания были применены на практике для построения проектов баз данных, физические и логические модели баз данных. В последствии данные базы данных были реализованы в СУБД MySQL. Также было создано пользовательское веб~--~приложение, представляющее удобный интерфейс для взаимодействия с информационной системой и базой данных.

%%% Local Variables: 
%%% mode: latex
%%% TeX-master: "rpz"
%%% End: 


% % Список литературы при помощи BibTeX
% Юзать так:
%
% pdflatex rpz
% bibtex rpz
% pdflatex rpz

\bibliographystyle{gost780u}
\bibliography{course-plis/rpz}

%%% Local Variables: 
%%% mode: latex
%%% TeX-master: "rpz"
%%% End: 


\appendix   % Тут идут приложения
%
%\include{90-appendix1}
%\include{91-appendix2}

\end{document}

%%% Local Variables:
%%% mode: latex
%%% TeX-master: t
%%% End:
