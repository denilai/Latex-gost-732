%% Преамбула TeX-файла

% 1. Стиль и язык
\documentclass[utf8x]{G7-32} % Стиль (по умолчанию будет 14pt)
\usepackage[T2A]{fontenc}
\usepackage[russian]{babel}
\setcounter{page}{4}
% Остальные стандартные настройки убраны в preamble.inc.tex.
%%%==========================================%%
%%% Для избежания переносов слов
%\usepackage{ragged2e}
%\usepackage{microtype}
%\usepackage{setspace}
%%\полуторный интервал
%\onehalfspacing
%\justifying
%\sloppy
%\tolerance=500
%\hyphenpenalty=10000
%\emergencystretch=3em
%%%==========================================%%

% Настройки стиля ГОСТ 7-32
% Для начала определяем, хотим мы или нет, чтобы рисунки и таблицы нумеровались в пределах раздела, или нам нужна сквозная нумерация.
\EqInChapter % формулы будут нумероваться в пределах раздела
\TableInChapter % таблицы будут нумероваться в пределах раздела
\PicInChapter % рисунки будут нумероваться в пределах раздела

% Добавляем гипертекстовое оглавление в PDF
\usepackage[
bookmarks=true, colorlinks=true, unicode=true,
urlcolor=black,linkcolor=black, anchorcolor=black,
citecolor=black, menucolor=black, filecolor=black,
]{hyperref}

%Величина «красной строки»
% Красная строка
\usepackage{indentfirst}
\setlength{\parindent}{1.25cm} % для шрифта 14pt

\makeatletter
\renewcommand{\@oddfoot}{\hfil \small \arabic{page}\hfil}
\renewcommand{\@evenfoot}{\hfil \small \arabic{page}\hfil}
\makeatother


\usepackage[tableposition=top,singlelinecheck=false]{caption}
%\usepackage{subcaption}


%подсчет сущностей

\usepackage{lastpage}

\usepackage{etoolbox}

\newcounter{totfigures}
\newcounter{tottables}
\newcounter{totreferences}
\newcounter{totappendix}
\pretocmd{\bibitem}{\addtocounter{totreferences}{1}}{}{}
\pretocmd{\chapter}{\addtocounter{totfigures}{\value{figure}}}{}{}
\pretocmd{\chapter}{\addtocounter{tottables}{\value{table}}}{}{}

\makeatletter
\AtEndDocument{%
	\addtocounter{totfigures}{\value{figure}}%
	\addtocounter{tottables}{\value{table}}%
	\immediate\write\@mainaux{%
		\string\gdef\string\totfig{\number\value{totfigures}}%
		\string\gdef\string\tottab{\number\value{tottables}}% 
		\string\gdef\string\totref{\number\value{totreferences}}%   
	}%
}
\makeatother



\DeclareCaptionLabelFormat{gostfigure}{Рисунок #2}
\DeclareCaptionLabelFormat{gosttable}{Таблица #2}
\DeclareCaptionLabelSeparator{gost}{~---~}
% Можно разбивать длинные таблицы вручную, оформляя каждую как table. В этом случае для продолжений таблицы нужно создать отдельный стиль заголовка
\DeclareCaptionLabelFormat{continued}{Продолжение таблицы~#2}
\captionsetup*[ContinuedFloat]{labelformat=continued}
\captionsetup{labelsep=gost}
\captionsetup*[figure]{labelformat=gostfigure, justification=centering,font={small, bf}}  % выравнивание по центру
\captionsetup*[table]{hangindent=-10pt, indention=0pt,parindent=0pt,margin=0pt,labelformat=gosttable,justification=raggedright,font={small,it}}
\captionsetup*[lstlisting]{font={small, it}}

% Изменение начертания шрифта --- после чего выглядит таймсоподобно.
% apt-get install scalable-cyrfonts-tex

\IfFileExists{cyrtimes.sty}
    {
        \usepackage{cyrtimespatched}
    }
    {
        % А если Times нету, то будет CM...
    }

\usepackage{graphicx}   % Пакет для включения рисунков
%расположение графики
\graphicspath{{images/}{figures/}{screnshots/}{../images/}}  % папки с картинками

% С такими оно полями оно работает по-умолчанию:
% \RequirePackage[left=20mm,right=10mm,top=20mm,bottom=20mm,headsep=0pt]{geometry}
% Если вас тошнит от поля в 10мм --- увеличивайте до 20-ти, ну и про переплёт не забывайте:
\geometry{right=10mm, left=30mm, top=20mm, bottom=20mm}
%\geometry{left=30mm}


% Пакет Tikz
\usepackage{tikz}
\usetikzlibrary{arrows,positioning,shadows}

% Произвольная нумерация списков.
\usepackage{enumerate}

% ячейки в несколько строчек
\usepackage{multirow}

% itemize внутри tabular
\usepackage{paralist,array}


% Настройки листингов.
\include{listings.inc}

% Полезные макросы листингов.
\include{macros.inc}

\renewcommand{\rmdefault}{ftm}

\begin{document}

 % выключает нумерацию ВСЕГО; здесь начинаются ненумерованные главы: реферат, введение, глоссарий, сокращения и прочее.

% Команды \breakingbeforechapters и \nonbreakingbeforechapters
% управляют разрывом страницы перед главами.
% По-умолчанию страница разрывается.

% \nobreakingbeforechapters
% \breakingbeforechapters

%% Также можно использовать \Referat, как в оригинале
\begin{abstract}
	

	\newcounter{totfigures}
	\newcounter{tottables}
	\makeatletter
	\AtEndDocument{%
		\addtocounter{totfigures}{\value{figure}}%
		\addtocounter{tottables}{\value{table}}%
		\immediate\write\@mainaux{%
			\string\gdef\string\totfig{\number\value{totfigures}}%
			\string\gdef\string\tottab{\number\value{tottables}}%    
		}%
	}
	\makeatother
	
	
	\pretocmd{\chapter}{\addtocounter{totfigures}{\value{figure}}}{}{}
	\pretocmd{\chapter}{\addtocounter{tottables}{\value{table}}}{}{}
	% ...
	Дипломная работа: \pageref*{LastPage}~с., \totfig~рис., \tottab~табл...
	

	
Это пример каркаса расчётно-пояснительной записки, желательный к использованию в РПЗ проекта по курсу РСОИ.

Данный опус, как и более новые версии этого документа, можно взять по адресу (\url{https://github.com/rominf/latex-g7-32}).

Текст в документе носит совершенно абстрактный характер.
\end{abstract}

%%% Local Variables: 
%%% mode: latex
%%% TeX-master: "rpz"
%%% End: 

\thispagestyle{empty}
\tableofcontents
\frontmatter

%\include{10-defines}
\Abbreviations %% Список обозначений и сокращений в тексте
% по алфавиту от англ к русским
\begin{description}
\item [CDC] Clock domain crossing. Пересечение доменов синхрочастоты
\item [CLI] Command line interface
\item [DDL] Data Definition Language
\item [DML] Data Manipulation Language
\item [CSS] Cascading Style Sheets
\item [HTML] HyperText Markup Language
\item[SQL] Structured query language.
\item[БД] База данных.
\item[ПО] Программное обеспечение.
\item[СУБД] Система управления базами данных.

\end{description}

%%% Local Variables:
%%% mode: latex
%%% TeX-master: "rpz"
%%% End:


\Introduction

Целью работы является построение цифрового устройства ресинхронизации данных. Данное устройство представляет собой FIFO с восьмиразрядным входом и тридцатидвух разрядным выходом. Данные на выход подаются по мере образования во входном буфере не менее 4 8-разрядных слов. Тактовые сигналы входного и выходного каналов независимы, однако тактовая частота для выхода FIFO составляет как минимум  четверть от тактовой частоты входного интерфейса FIFO.


%\begin{itemize}
%\item проанализировать предложенную ;
%\item спроектировать свою, новую всячину;
%\item изготовить всякую всячину;
%\item проверить её работоспособность.
%\end{itemize}
%
%Вот так-то. А этот абзац вставлен для визуальной оценки отступа от перечня до следующего абзаца.

\mainmatter % это включает нумерацию глав и секций в документе ниже

\newcommand{\erassistant}{ErAssistant~}

\chapter{ПРОБЛЕМАТИКА РАЗРАБОТКИ \\МУЛЬТИСИНХРОННЫХ УСТРОЙСТВ}

В данном разделе будут рассмотрены проблемы и особенности построение устройств


В этом разделе будут описаны проблемы, возникающие в процессе разработки мультисихронного проекта, то есть устройства, в котором имеют место пересечения клоковых доменов или доменов синхрочастоты (CDC).

\section{Домен синхрочастоты}
Домен синхрочастоты представляет собой ту часть проекта, которая тактируется одной или несколькими синхрочастотами, причем все эти синхрочастоты должны иметь постоянные сдвиги фазы. Если в какой-либо части проекта имеется синхрочастота или инвертированная синхрочастота, или синхрочастота, полученная из исходной путем деления на 2, то такая часть проекта считается клоковым доменом с одной синхрочастотой. Если же домены имеют синхрочастоты с переменной фазой и соотношениями времени, то такие домены считают доменами с различными синхрочастотами. 

На Рисунке \ref{fig:clock-domain} показано, что проект имеет единственный домен синхрочастоты, потому что синхрочастота divClk --- есть деленная на два частота генератора синхронизации Clk.

\begin{figure}
	\centering
	\includegraphics[width=0.6\linewidth]{course-scheme/images/clock-domain}
	\caption{Проект, имеющий единственный домен синхрочастоты}
	\label{fig:clock-domain}
\end{figure}


На Рисунке \ref{fig:multiclock-domain} показано несколько синхрочастот от различных источников. Ту часть проекта, которая управляется этими синхрочастотами, называют доменами синхрочастоты, и сигналы, осуществляющие передачу импульсов между этими асинхронными доменами синхрочастоты, называют путями пересечения домена синхрочастоты. Сигнал DA считают асинхронным сигналом в домене синхрочастоты, так как между генератором синхронизации A (clkA) и генератором синхронизации B (ClkB) не существуют постоянные соотношения фазы и времени.

\begin{figure}
	\centering
	\includegraphics[width=0.7\linewidth]{course-scheme/images/multiclock-domain}
	\caption{Путь домена синхрочастоты}
	\label{fig:multiclock-domain}
\end{figure}


\section{Основные принципы}

При разработке ультисинхрочастотных проектов следует уделять особое внимание стабильности сигнала. Когда сигнал пересекает домен синхрочастоты, то он появляется в новом домене синхрочастоты как асинхронный сигнал и должен быть засинхронизирован.

Синхронизация предотвращает в новом домене синхрочастоты метастабильное состояние первого запоминающего элемента схемы (триггера), и это позволяет в новом домене работать уже со стабильным сигналом.

\textit{Метастабильность} --- это неспособность триггера достигнуть известного состояния в определенный момент времени. Когда триггер входит в метастабильное состояние, то невозможно предсказать ни уровень выходного напряжения элемента, ни период времени, за который этот выход перейдет к правильному уровню напряжения. 

В течение этого переходного времени выход триггера будет находиться на некотором промежуточном уровне напряжения или колебаться и может передать этот недопустимый уровень сигнала со своего выхода к другим триггерам схемы.

\section{Решение проблемы метастабильности}

С целью решения проблемы метастабильности применяются каскады стабилизирующих триггеров, включенных последовательно. Рассмотрим это решение.


Простейший синхронизатор представляет собой два триггера, включенных последовательно без какой-либо комбинационной схемы между ними. Такая схема проекта гарантирует, что первый триггер выходит из своего метастабильного состояния, и его выход переходит в устойчивое состояние перед тем, как второй триггер сохраняет его. Необходимо также разместить эти триггеры, насколько возможно, ближе друг к другу для того, чтобы гарантировать наименьшую расфазировку тактовых сигналов между ними. 

Другой тип ячейки синхронизатора представляет собой два близко расположенных триггера без какой-либо комбинационной логики между ними. Для того чтобы синхронизация работала должным образом, сигнал, пересекающий домен синхрочастоты, должен проследовать от триггера в домене синхрочастоты источника сигнала к первому триггеру синхронизатора, не пройдя через комбинационную логику \ref{fig:sync-triggers}). Схема синхронизации приведена на Рисунке \ref {fig:sync-scheme}.


\begin{figure}
	\centering
	\includegraphics[width=0.7\linewidth]{course-scheme/images/sync-triggers}
	\caption{Триггеры синхронизации}
	\label{fig:sync-triggers}
\end{figure}

\begin{figure}
	\centering
	\includegraphics[width=0.5\linewidth]{course-scheme/images/sync-scheme}
	\caption{Схема синхронизации сигнала}
	\label{fig:sync-scheme}
\end{figure}



Данный способ борьбы с метастабильностью будет в дальнейшем использован при разработке узла ресихронизации данных.

\clearpage
\chapter{ПРОЕКТИРОВАНИЕ ЦИФРОВОГО УСТРОЙСТВА}

\label{sec:design}
В данном разделе освещается разработка функциональной и структурной схем устройства по ресинхронизации данных. Определим структурные элементы устройства, принцип и порядок его работы.

\section{Организация взаимодействия с устройством}
Взаимодействие с устройством производится посредством передачи восьмибитного слова на вход wdata. Аппаратный блок осуществляет запись данных во внутреннюю память по сигналам тактового генератора домена источника данных wclk, основываясь на состоянии входных сигналов wrst, wen и служебных сигналов wr\_full,  wptr, rptr.

Чтение из устройства ресинхронизации производится посредством считывания тридцатидвухразрядного слова, составляемого из четырех последовательно поданных на вход восьмиразрядных слов в порядке \textit{от старшего к младшему} (англ. big-endian), генерируемого на выходе rdata. Аппаратный блок осуществляет чтение данных из внутренней памяти по сигналам тактового генератора домена получателя данных, основываясь на состоянии входных сигналов rrst, ren  r\_empty wptr, rptr, rdredy.

\section{Организация памяти}

Устройство хранит данные в собственной памяти. Устройство может одновременно хранить шестнадцать восьмибитных слов, используя для их адресации пятиразрядный адрес, в котором старший бит призван реализовать возможность контроля заполнения и опустошения очереди. 

Возможность асинхронного сброса памяти не реализована в силу того, что обычно память в устройствах, реализуемых на ПЛИС представлена в виде блочной оперативной памяти и выделенной оперативной памяти. Эти ресурсы обычно не поддерживают сброс. Сброс возможно реализовать в регистрах. 



\section{Тактирование}
Управление устройством производится по двух синхросигналам wclk и rclk. С целью оптимальной работы цифрового устройства при передаче данных из одного домена синхросигнала в другой тактовые сигналы входного и выходного каналов сделаны независимыми, однако тактовая частота для выхода очереди составляет как минимум $\dfrac14$ от тактовой частоты входного интерфейса FIFO.

Для корректной передачи и предотвращения потери данных, а также с целью борьбы с метастабильностью в случае независимых тактовых сигналов используются группа синхронизирующих триггеров, которые передают текущее состояние адреса чтения и адреса записи между клоковыми доменами. Основываясь на значениях данных адресов делается вывод о заполненности и пустоте очереди. 

\section{Адресация}

Адресация в памяти реализована с помощью двух пятиразрядных переменных, хранящих адрес ячеек,  с которыми будут произведены операции чтения и записи при следующем синхросигнале.

Важной частью работы устройства является работа с памятью. Необходимо избежать потери и перезаписи не переданных данных. С этой целью на каждом шаге работы схемы осуществляется проверка и сравнивание адресов чтения и записи.

Во время работы схема хранит адреса в виде рефлексивного двоичного кода Грея. Выбор в сторону кода Грея сделан для снижения критичности ошибки определения адреса чтения или записи при его передаче. 

В случае использования кода Грея при переходе к следующему адресу изменяется только один бит, и даже если схема не сможет верно определить значение адреса~---~посчитает, что адреса не изменился, это не приведет к нарушению работы системы. 

В то время как при использовании прямого двоичного кода ошибка синхронизации неизбежна. При переходе могут изменяться все биты адреса (например при переходе $01111 \to 1000$). В данном случае значение адреса является полностью неопределенным. Это может привести к нарушению в работе устройства в целом.

Когда указатели чтения и записи равны, это указывает на то, что FIFO пуст. Эта ситуация возникает во время операции сброса или когда было произведено чтение последнего слова в FIFO, как показано на Рисунке \ref{fig:empty-fifo}.

\begin{figure}[h!]
	\centering
	\includegraphics[width=0.7\linewidth]{course-scheme/images/empty-fifo}
	\caption{Все данные прочитаны. Очередь пуста}
	\label{fig:empty-fifo}
\end{figure}

Когда указатели чтения и записи снова равны, это означает, что очередь заполнена и дальнейшая запись не может быть произведена. Эта ситуация возникает, когда переменная адреса записи переполняется, как показано на Рисунке \ref{fig:full-fifo}.

\begin{figure}[h!]
	\centering
	\includegraphics[width=0.7\linewidth]{course-scheme/images/full-fifo}
	\caption{Очередь заполнена. Запись не допустима}
	\label{fig:full-fifo}
\end{figure}

В обоих случаях адреса записи и чтения указывают на одну и ту же ячейку памяти, но при этом ситуации являются противоположными, что усложняет логику определения текущего состояния памяти цифрового устройства.

С целью определения текущего состояния асинхронной очереди необходимо использовать на один бит больше, чем требуется для адресации всех ячеек запоминающего устройства. Сравнение этого бита даст ответ о состоянии памяти в случае, когда указатели чтения и записи окажутся равны.

Пятиразрядные сигналы wr\_full и rd\_empty, позволяющие организовать корректную работу с устройством, формируются с помощью следующих логических условий.
\begin{align}
	 wr\_full = &\left(wptr  == \left\{!rptr[4:3],rptr[aw-2:0]\right\}\right);\\
	 &rd\_empty = \left(wptr  == rptr\right)
\end{align}

В устройстве необходимо реализовать проверку обращения к одному элементу памяти. Чтение по данному адресу должно быть заблокировано до тех пор, пока адрес записи не перестанет указывать на данную ячейку памяти.

При каждом приходящем тактовом сигнале устройство осуществляет сравнение адресов чтения и записи, предварительно производя с ними промежуточные действия. Приведем список преобразований и производимых операций.

\begin{itemize}
	\item перевод из прямого двоичного кода в рефлексивный код Грея;
	\item сохранение текущего значения в регистре;
	\item передача значения адреса в синхронизационные регистры;
	\item перевод из рефлексивного кода Грея в прямой двоичный код;
	\item синхронизация приходящего адреса генератором соседнего домена синхронизации;
	\item сравнение адресов чтения и записи.
\end{itemize}

Данные операции производятся с адресом чтения и записи отдельно. Порядок следования этих операций можно пронаблюдать на Рисунке \ref{fig:signal-process}.

\begin{figure}
	\centering
	\includegraphics[width=1\linewidth]{course-scheme/images/signal-process}
	\caption{Обработка адресов чтения и записи}
	\label{fig:signal-process}
\end{figure}



\section{Функциональная схема}
Предлагается для построения устройства ресинхронизации данных, работающего в режиме очереди с независимыми тактовыми сигналами, использовать данную функциональную схему (см. Рисунок \ref{fig:unit-design}).

\begin{figure}[h!]
	\centering
	\includegraphics[width=1\linewidth]{course-scheme/images/unit-design}
	\caption{Функциональная схема устройства}
	\label{fig:unit-design}
\end{figure}


Приведем описание входов, выходов и служебных сигналов устройства (см. Таблицу \ref{tab:sig-ports}).

Также укажем используемые глобальные и локальные параметры, используемые при реализации устройства на языке описания аппаратуры Verilog (см. Таблицу \ref{tab:params}). С помощью данных параметров возможно произвести масштабирование устройства --- увеличить разрядность ячейки запоминающего устройства и их количество.

\begin{table}[htbp]
	\centering
	\fontsize{12}{16pt}\selectfont
	\caption{Параметры}
	\begin{tabular}{|l|p{0.5\linewidth}|c|}
		\hline
		\multicolumn{1}{|c}{\textbf{Название}} & \multicolumn{1}{|c|}{\textbf{Назначение}} & \multicolumn{1}{c|}{\textbf{Значение}} \\ \hline
		asize  &  разрядность адреса данных & 4 \\ \hline
		dsize  &  разрядность данных в битах & 8 \\ \hline
	\end{tabular}
	\label{tab:params}
\end{table}

Аппаратный блок по ресинхронизации данных разрабатывается для использования в совокупности с более сложными элементами и выполняет роль служебного блока хранения и передачи данных.

И



\begin{table}[h!]
	\fontsize{12}{16pt}\selectfont
	\centering
	\caption{Сигналы входы и выходы устройства}
	\label{tab:sig-ports}
	\begin{tabular}{|l|l|c|}
		\hline
		\multicolumn{1}{|c}{\textbf{Название}} & \multicolumn{1}{|c|}{\textbf{Назначение}} & \multicolumn{1}{c|}{\textbf{Разрядность}} \\ \hline
		mem  &  ячейки памяти очереди & [8:0] [0:15] \\ \hline
		rbin  &  текущий адрес чтения в бинарном коде & [4:0] \\ \hline
		rbinnext  &  адрес чтения в бинарном коде в следующем такте & [4:0] \\ \hline
		rclk  &  тактирование на чтение данных & 1 \\ \hline
		rdata  &  данные на чтение & [31:0] \\ \hline
		rden  &  разрешение на чтение & 1 \\ \hline
		rdredy  &  готовность выдать данные & 1 \\ \hline
		rempty  &  сигнал пустоты очереди (прочтено все, что записано) & 1 \\ \hline
		rempty\_next  &  пустота очереди на следующем такте & 1 \\ \hline
		rgray  &  текущий адрес чтения в коде Грея & [4:0] \\ \hline
		rgraynext  &  адрес чтения в коде Грея в следующем такте & [4:0] \\ \hline
		rq1\_wgray  &  адрес чтения в первом синхро-триггере & [4:0] \\ \hline
		rq2\_wgray  &  адрес чтения во втором синхро-триггере & [4:0] \\ \hline
		rrstn  &  сброс указателя чтения & 1 \\ \hline
		wbin  &  текущий адрес записи в бинарном коде & 1 \\ \hline
		wbinnext  &  адрес записи в бинарном коде в следующем такте & [4:0] \\ \hline
		wclk  &  Тактирование на запись данных & 1 \\ \hline
		wdata  &  данные на запись & [7:0] \\ \hline
		wfull  &  сигнал заполнения очереди & 1 \\ \hline
		wfull\_next  &  заполненность очереди на следующем такте & 1 \\ \hline
		wgray  &  текущий адрес записи в коде Грея & [4:0] \\ \hline
		wgraynext  &  адрес записи в коде Грея в следующем такте & [4:0] \\ \hline
		wq1\_rgray  &  адрес записи в первом синхро-триггере & [4:0] \\ \hline
		wq2\_rgray  &  адрес записи во втором синхро-триггере & [4:0] \\ \hline
		wren  &  разрешение на запись & 1 \\ \hline
		wrstn  &  сброc указателя записи & 1 \\ \hline
	\end{tabular}
\end{table}
\normalsize
\clearpage







\clearpage
\clearpage
\chapter{Создание базы данных}
\section{Работа по методическим указаниям}
Создадим базу данных \texttt{forum}, которая хранит в себе сведения о пользователях форумах и размещенных ими темах.

Помимо суперпользователя root, был создан пользователь denilai, под которым производятся все манипуляции с данными.

Создадим базу данных \texttt{forum} с помощью команды \texttt{CREATE DATABASE forum;}


Создадим таблицу \texttt{users} (см. Рисунок \ref{fig:create-users}):

% TODO: \usepackage{graphicx} required
\begin{figure}[h!]
	\centering
	\includegraphics[width=0.7\linewidth]{create-users}
	\caption{Создание базы данных users}
	\label{fig:create-users}
\end{figure}

%\begin{lstlisting}
%create table users (
%	id_user INT (10) AUTO_INCREMENT,
%	name varchar(20) NOT NULL,
%	email varchar(50) NOT NULL,
%	password varchar(15) NOT NULL,
%	PRIMARY KEY (id_user)
%);
%\end{lstlisting}

Создадим таблицу \texttt{topics} (см. Рисунок \ref{fig:create-topics}):

% TODO: \usepackage{graphicx} required
\begin{figure}[h!]
	\centering
	\includegraphics[width=0.7\linewidth]{create-topics}
	\caption{Создание базы данных topics}
	\label{fig:create-topics}
\end{figure}

%\begin{lstlisting}
%create table topics (
%	id_topic INT (10) AUTO_INCREMENT,
%	topic_name varchar(100) NOT NULL,
%	id_author INT (10) NOT NULL,
%	PRIMARY KEY (id_topic),
%	FOREIGN KEY (id_author) REFERENCES users (id_user)
%);
%\end{lstlisting}
Создадим таблицу \texttt{posts} (см. Рисунок \ref{fig:create-posts}):

\begin{figure}[h!]
	\centering
	\includegraphics[width=0.7\linewidth]{create-posts}
	\caption{Создание базы данных posts}
	\label{fig:create-posts}
\end{figure}

После создания таблиц, заполним их данными о пользователях форума, о темах и размещенных публикациях. Выполним операцию выборки данных без условия, чтобы увидеть все записи, занесенные в таблицы с помощью команды \texttt{SELECT * FROM <table-name>} (см. Рисунок \ref{fig:select-from-all}). 

% TODO: \usepackage{graphicx} required
\begin{figure}[h!]
	\centering
	\includegraphics[width=0.8\linewidth]{select-from-all}
	\caption{Операция выборки из всех таблиц }
	\label{fig:select-from-all}
\end{figure}

Выполним запрос \texttt{SELECT mesage, topic\_name FROM posts p JOIN topics t ON t.id\_author = p.id\_author; } для объединения данных из таблиц \texttt{topics} и \texttt{posts} по ключу id\_author и получения полной информации о сообщении и названию темы, в которой оно было размещено (см. Рисунок \ref{fig:join-post-author}).

% TODO: \usepackage{graphicx} required
\begin{figure}[ht]
	\centering
	\includegraphics[width=0.8\linewidth]{join-post-author}
	\caption{Запрос объединения}
	\label{fig:join-post-author}
\end{figure}

Выполним запрос выборки данных, явно указав поля отношения. Для этого перечислим имена полей через запятую после зарезервированного слова \texttt{SELECT} (см. Рисунок \ref{fig:field-select}).

% TODO: \usepackage{graphicx} required
\begin{figure}[h!]
	\centering
	\includegraphics[width=0.5\linewidth]{field-select}
	\caption{Операция выборки с указанием полей}
	\label{fig:field-select}
\end{figure}


Выполним более сложные запросы выборки, отсортировав записи в таблице \texttt{topics} по убыванию значения поля \texttt{topic\_name} и \texttt{id\_author}, а также опишем условие сравнения значения поля \texttt{id\_author} в сецкции \texttt{WHERE}(см. Рисунок \ref{fig:order-where-forum}). 

% TODO: \usepackage{graphicx} required
\begin{figure}[h!]
	\centering
	\includegraphics[width=0.5\linewidth]{order-where-forum}
	\caption{Сложные запросы выборки с сортировкой и условием}
	\label{fig:order-where-forum}
\end{figure}

Выполним операции по модификации таблицы --- добавим в таблицу \texttt{users} поле \texttt{country} типа \texttt{varchar (20)} со значением по умолчанию "Russia", а также добавим в эту же таблицу поле \texttt{age int(10)} со значением по умолчанию 19. 

Выведем все записи из таблицы \texttt{users}, обнаружим, что столбцы были вставлены успешно (см. Рисунок \ref{fig:alter-add-forum}).

% TODO: \usepackage{graphicx} required
\begin{figure}[h!]
	\centering
	\includegraphics[width=0.9\linewidth]{alter-add-forum}
	\caption{Добавление полей в таблицу \texttt{users}}
	\label{fig:alter-add-forum}
\end{figure}

В ходе данной практической работы были рассмотрены операторы DDL и DML диалекта MySQL. С помощью данных операторов была создана база учебная база данных \texttt{forum}, содержащая о пользователях форумах и размещенных ими темах.

\section{Индивидуальное задание}

Продолжим работу над созданием модели данных велосипедного предприятия. Создадим базу данных \texttt{cycle}, сущностями  которой будут таблицы, описания которых были проработаны в прошлых практических работах. 

Описание проектируемых отношений базы данных \texttt{cycle} приведены в таблицах \ref{tab:bar}--\ref{tab:log}.

%(\textbf{Bar})
\begin{table}[h!] 
	\centering
	\caption{Описание таблицы Bar}
	\begin{tabular}{|l|l|}
		\hline \textbf{Имя} & \textbf{Тип} \\
		\hline
		Width & INTEGER NOT NULL \\ \hline
		Diameter & INTEGER NOT NULL \\ \hline
		BarID & INTEGER NOT NULL AUTO\_INCREMENT \\ \hline
		Color & VARCHAR(20) NULL DEFAULT 'black' \\ \hline
	\end{tabular}
	\label{tab:bar}
\end{table}

%(\textbf{Brake})
\begin{table}[h!] 
	\centering
	\caption{Описание таблицы Brake}
	\begin{tabular}{|l|l|}
		\hline \textbf{Имя} & \textbf{Тип} \\
		\hline
		BrakeID &       INT NOT NULL AUTO\_INCREMENT \\ \hline
		ComponentID &   INTEGER NULL \\ \hline
	\end{tabular}
	\label{tab:brake}
\end{table}


%(\textbf{Component})

\begin{table}[h!] 
	\centering
	\caption{Описание таблицы Component}
	\begin{tabular}{|l|l|}
		\hline \textbf{Имя} & \textbf{Тип} \\
		\hline
		ComponentID &   INTEGER NOT NULL AUTO\_INCREMENT \\ \hline
		Brand & VARCHAR(20) NOT NULL \\ \hline
		ManufacturerCountry &  VARCHAR(20) NOT NULL \\ \hline
		PriceCent & INTEGER NOT NULL \\ \hline
		Model & VARCHAR(20) NOT NULL \\ \hline
		ComponentTypeID &  INTEGER NOT NULL \\ \hline
		WeightGramm &   INTEGER NOT NULL \\ \hline
	\end{tabular}
	\label{tab:component}
\end{table}

%(\textbf{ComponentType}) 
\begin{table}[h!] 
	\centering
	\caption{Описание таблицы ComponentType}
	\begin{tabular}{|l|l|}
		\hline \textbf{Имя} & \textbf{Тип} \\
		\hline
		ComponentTypeID &  INTEGER NOT NULL AUTO\_INCREMENT \\ \hline
		Name & VARCHAR(20) \\ \hline
	\end{tabular}
	\label{tab:componenttype}
\end{table}


%(\textbf{CycleType})

\begin{table}[h!] 
	\centering
	\caption{Описание таблицы CycleType}
	\begin{tabular}{|l|l|}
		\hline \textbf{Имя} & \textbf{Тип} \\
		\hline
		CycleTypeID &   INTEGER NOT NULL AUTO\_INCREMENT \\ \hline
		Name &  VARCHAR(20) NOT NULL \\ \hline
	\end{tabular}
	\label{tab:cycletype}
\end{table}

\begin{table}[h!] 
	\centering
	\caption{Описание таблицы FctCycleBuild}
	\begin{tabular}{|l|l|}
		\hline \textbf{Имя} & \textbf{Тип} \\
		\hline
		SetupID &       INTEGER NOT NULL \\ \hline
		Datetime &      DATE NOT NULL \\ \hline
		Stage & INTEGER NOT NULL \\ \hline
		Workshop &      varchar(20) NOT NULL DEFAULT 'main' \\ \hline
		Master &        VARCHAR(20) NOT NULL \\ \hline
		Status &        VARCHAR(20) NOT NULL \\ \hline
		Comment &       VARCHAR(20) NOT NULL \\ \hline
	\end{tabular}
	\label{}
\end{table}

%(\textbf{Fork})

\begin{table}[h!] 
	\centering
	\caption{Описание таблицы Fork}
	\begin{tabular}{|l|l|}
		\hline \textbf{Имя} & \textbf{Тип} \\
		\hline
		ForkID &  INTEGER NOT NULL AUTO\_INCREMENT \\ \hline
		Color & VARCHAR(20) NULL DEFAULT 'black' \\ \hline
		ComponentID &  INTEGER NULL \\ \hline
	\end{tabular}
	\label{ta:fork}
\end{table}


%(\textbf{Frame})

\begin{table}[h!] 
	\centering
	\caption{Описание таблицы Frame}
	\begin{tabular}{|l|l|}
		\hline \textbf{Имя} & \textbf{Тип} \\
		\hline
		FrameID &  INTEGER NOT NULL AUTO\_INCREMENT \\ \hline
		Color & VARCHAR(20) NULL DEFAULT 'black' \\ \hline
	\end{tabular}
	\label{tab:frame}
\end{table}


%(\textbf{FrameInfo})

\begin{table}[h!] 
	\centering
	\caption{Описание таблицы FrameInfo}
	\begin{tabular}{|l|l|}
		\hline \textbf{Имя} & \textbf{Тип} \\
		\hline
		FrameID &       INTEGER NOT NULL \\ \hline
		Size & INTEGER NOT NULL \\ \hline
		Stack & INTEGER NOT NULL \\ \hline
		Reach & INTEGER NOT NULL \\ \hline
	\end{tabular}
	\label{tab:frameinfo}
\end{table}


%(\textbf{Frameset})

\begin{table}[h!] 
	\centering
	\caption{Описание таблицы Frameset}
	\begin{tabular}{|l|l|}
		\hline \textbf{Имя} & \textbf{Тип} \\
		\hline
		FrameSizeID &   INTEGER NOT NULL AUTO\_INCREMENT \\ \hline
		ForkID &        INTEGER NOT NULL \\ \hline
		FramesetID &    INTEGER NOT NULL \\ \hline
		ComponentID &   INTEGER NOT NULL \\ \hline
	\end{tabular}
	\label{tab:frameset}
\end{table}


%(\textbf{FrameSize})

\begin{table}[h!] 
	\centering
	\caption{Описание таблицы FrameSize}
	\begin{tabular}{|l|l|}
		\hline \textbf{Имя} & \textbf{Тип} \\
		\hline
		FrameID &       INTEGER NOT NULL \\ \hline
		Size &  INTEGER NOT NULL \\ \hline
		FrameSizeID &   INTEGER NOT NULL AUTO\_INCREMENT \\ \hline
		ComponentID &   INTEGER NOT NULL \\ \hline
	\end{tabular}
	\label{tab:framesize}
\end{table}


%(%(\textbf{Groupset}))

\begin{table}[h!] 
	\centering
	\caption{Описание таблицы Groupset}
	\begin{tabular}{|l|l|}
		\hline \textbf{Имя} & \textbf{Тип} \\
		\hline
		Ratio & INTEGER NOT NULL \\ \hline
		BrakeID &       CHAR(18) NOT NULL \\ \hline
		GroupsetID &    INTEGER NOT NULL AUTO\_INCREMENT \\ \hline
		Color & VARCHAR(20) NOT NULL \\ \hline
		ComponentID &   INTEGER NOT NULL \\ \hline
	\end{tabular}
	\label{tab:groupset}
\end{table}


%(\textbf{Setup})

\begin{table}[h!] 
	\centering
	\caption{Описание таблицы Setup}
	\begin{tabular}{|l|l|}
		\hline \textbf{Имя} & \textbf{Тип} \\
		\hline
		GroupsetID &    INTEGER NOT NULL \\ \hline
		WheelsetID &    INTEGER NOT NULL \\ \hline
		FramesetID &    INTEGER NOT NULL \\ \hline
		BarID & INTEGER NOT NULL \\ \hline
		CycleTypeID &   INTEGER NOT NULL \\ \hline
		SetupID &       INTEGER NOT NULL AUTO\_INCREMENT \\ \hline
	\end{tabular}
	\label{tab:setup}
\end{table}


%(\textbf{Wheelset})

\begin{table}[h!] 
	\centering
	\caption{Описание таблицы Wheelset}
	\begin{tabular}{|l|l|}
		\hline \textbf{Имя} & \textbf{Тип} \\
		\hline
		DiametrInch &   INTEGER NOT NULL \\ \hline
		Width & INTEGER NOT NULL \\ \hline
		SpokesCount &   INTEGER NULL \\ \hline
		WheelsetID &    INTEGER NOT NULL AUTO\_INCREMENT \\ \hline
		Color & VARCHAR(20) NOT NULL \\ \hline
		ComponentID &   INTEGER NOT NULL \\ \hline
	\end{tabular}
	\label{tab:wheelset}
\end{table}

\begin{table}[h!] 
	\centering
	\caption{Описание таблицы Log}
	\begin{tabular}{|l|l|}
		\hline \textbf{Имя} & \textbf{Тип} \\
		\hline
		ID &   INTEGER NOT NULL AUTO\_INCREMENT \\ \hline
		msg & VARCHAR{100} NOT NULL \\ \hline
		row\_id & INTEGER NOT NULL \\ \hline
	\end{tabular}
	\label{tab:log}
\end{table}



\newpage\hfill\newpage\hfill\newpage

\subsection{Использование MySQL CLI Client}

Создадим данные таблицы с помощью MySQL CLI Client~---~клиента, предоставляющего доступ к СУБД через интерфейс командой строки, использовав ключевое слово \texttt{CREATE TABLE}. После создания таблиц выполним команду \texttt{SHOW TABLES}, выбрав базу данных \texttt{cycle} (см. Рисунок \ref{fig:show-tables-cycle}).

% TODO: \usepackage{graphicx} required
\begin{figure}[h!]
	\centering
	\includegraphics[width=0.5\linewidth]{show-tables-cycle}
	\caption{таблицы базы данных \texttt{cycle}}
	\label{fig:show-tables-cycle}
\end{figure}
Выведем записи из таблицы \texttt{Component} (см. Рисунок \ref{fig:select-components}).

% TODO: \usepackage{graphicx} required
\begin{figure}[h!]
	\centering
	\includegraphics[width=\linewidth]{select-components}
	\caption{Вывод записей из таблицы Components}
	\label{fig:select-components}
\end{figure}

Выполним более сложные запросы выборки --- объединим таблицы с помощью оператора \texttt{INNER JOIN} \texttt{Component} и \texttt{fork} по полю \linebreak \texttt{component\_id}, а также воспользуемся функцией \texttt{ROW\_NUMBER()} в сочетании с оконной функцией \texttt{OVER()}, пронумеровав компоненты из таблицы \texttt{Fork} одного цвета по возрастанию значения поля \texttt{color} (см. Рисунок \ref{fig:join-window-cycle}).

% TODO: \usepackage{graphicx} required
\begin{figure}[h!]
	\centering
	\includegraphics[width=\linewidth]{join-window-cycle}
	\caption{Сложные запросы выборки. База данных cycle}
	\label{fig:join-window-cycle}
\end{figure}


Создадим триггеры \texttt{delete\_component} и  \texttt{add\_component} на добавление и удаление записи в таблице \texttt{Component}. При срабатываении данного триггера, будет добавляться запись в таблицу log, информирующая о совершении манипуляций с данными (см. Листинг \ref{list:trigger}).
%\newpage
Приведем объявление данного триггера на диалекте MySQL:
\lstset{
	language=sql
}
\begin{lstlisting}[caption=Триггер, label={list:trigger}]
DELIMITER $$
drop trigger delete_component; $$

create trigger `delete_component` after delete on component
for each row begin
insert into log (msg, row_id) values (concat('delete component ',old.Brand,' ', old.Model), old.ComponentID);
end; $$

create trigger `add_component` after insert on component
for each row begin
insert into log (msg, row_id) values (concat('insert component ',new.Brand,' ', new.Model), new.ComponentID);
end; $$
DELIMITER $$
\end{lstlisting}

\subsection{Использование MySQL Workbench}

Выполним операции по изменению и просмотру данных, занесенных в базу данных \texttt{cycle}, с помощью инструмента для визуального проектирования баз данных MySQL Workbench.

Для этого подключимся к локально развернутому на машине MySQL Server, указав порт, имя пользователя и пароль. После этого мы получим доступ к пользовательскому интерфейсу программы, представляющему собой две области~---~область выполнения запросов и область отображения результатов (см. Рисунок \ref{fig:workbrench-interface}).

\begin{figure}[h!]
	\centering
	\includegraphics[width=\linewidth]{figures/web-clent/workbrench-interface}
	\caption{Интерфейс MySQL Workbench}
	\label{fig:workbrench-interface}
\end{figure}

Создадим временную таблицу, в котором объединим сведения о компонентах (см. Рисунок \ref{fig:workbrench-components}).
\begin{figure}[h!]
	\centering
	\includegraphics[width=1\linewidth]{figures/web-clent/workbrench-components}
	\caption{Просмотр временной таблицы компонентов}
	\label{fig:workbrench-components}
\end{figure}

Выполним еще один запрос~---~ограничим вывод только компонентами типа <<Fork>> (см. Рисунок \ref{fig:workbrench-forks}).

\begin{figure}[h!]
	\centering
	\includegraphics[width=1\linewidth]{figures/web-clent/workbrench-forks}
	\caption{Просмотр временной таблицы компонентов. Компоненты <<Fork>>}
	\label{fig:workbrench-forks}
\end{figure}

Данный визуальный инструмент позволяет удобно организовать работу с базой данных, сохранять SQL--скрипты, параметры подключения и настройки.
\clearpage
\chapter{Создание веб--клиента}

В данном разделе будет рассмотрен процесс реализации веб--приложения, с помощью которого будет производится взаимодействие конечного пользователя с созданной в рамках предыдущих работ базой данных. 

\section{Общие требования к приложению}

Приведем технические требования, предъявляемые к разрабатываемому приложению.

\subsection{Требования к функционалу}
К функционалу приложения предъявляются следующие требования:
\begin{itemize}
	\item регистрация и авторизация пользователя в системе;
	\item добавление, изменение, удаление, обновление информации по
	теме;
	\item фильтрация списков по соответствующим признакам;
	\item просмотр информации по запросу.
\end{itemize}

\subsection{Требования к интерфейсу}

Интерфейс системы должен поддерживать русский язык.

Интерфейс системы должен быть спроектирован с учетом ролевой
модели и уровней доступа пользователей.

Интерфейс системы должен обеспечивать наглядное, интуитивно
понятное представление структуры размещенной информации, быстрый и
логичный переход к соответствующим разделам.

Навигационные элементы интерфейса должны обеспечивать понимание
пользователем их смысла и обеспечивать навигацию по всем доступным
пользователю разделам и отображать соответствующую информацию.

Интерфейс системы должен позволять решать задачи пользователя
наиболее быстрым, простым и удобным из возможных способов.

Дизайн и удобство интерфейса должны быть на уровне ожиданий
современного пользователя и восприниматься им как комфортная, удобная и
приятная рабочая среда.

\section{Реализация приложения}

На основании функциональных требований, указанных в задании, было реализовано веб приложение, позволяющие взаимодействовать с данными, хранящимися в базе данных \texttt{cycle}, созданной в ходе выполнения предыдущих практических работ данного курса.

Интерфейс системы спроектирован с учетом ролевой
модели и уровней доступа пользователей и подразумевает регистрацию пользователя в СУБД MySQL.

Интерфейс поддерживает русский язык и позволяет пользователю удобно просматривать сведения, содержащиеся в таблицах выбранной базы данных.

Акцент в приложении сделан на содержании, дизайн интерфейса прост и лаконичен, чтобы облегчить и упростить взаимодействие пользователя с системой.

\subsection{Используемые технологии}
В качестве набора программных технологий был выбран LAMP--стэк, объединяющий в себе следующие компоненты:
\begin{itemize}
	\item Linux;
	\item Apache;
	\item MySQL;
	\item PHP.
\end{itemize}

Программно-аппаратная часть сервиса (бэкенд) реализована на языке PHP в парадигме объектно-ориентированного программирования, а также с использованием языка MySQL для работы с базой данных.

Структура гипертекстового документа описана средствами HTML и CSS.

Клиентская сторона пользовательского интерфейса системы (фронтэнд) основана на подходе AJAX (Asynchronous Javascript and XML). В качестве языка программирования был использован Java Script. 

Запросы веб-сервиса обрабатываются сервером Apache. 

\subsection{Возможные сценария использования веб-приложения}

Реализуемое веб--приложение может быть использовано для просмотра таблиц, добавления данных в таблицы базы данных под управлением СУБД MySQL, а также для написания произвольных DML и DDL запросов на диалекте MySQL. 

Веб--приложение объединяет в себе часть возможностей командой строки СУБД MySQL и веб-клиента для просмотра таблиц.


\subsection{Демонстрация работы веб-приложения}

Продемонстрируем общие сценарий использования веб--приложения, созданного в рамках данной практической работы. 

\subsubsection{Авторизация}
При первом входе на сайт пользователю необходимо авторизоваться и выбрать пользователя СУБД MySQL, который имеет права на изменение и просмотр данных в выбранной базе данных. Форма авторизации приведена на Рисунке \ref{fig:auth-empty}.
\begin{figure}[h!]
	\centering
	\includegraphics[width=1\linewidth]{figures/web-clent/auth-empty}
	\caption{Авторизация пользователя}
	\label{fig:auth-empty}
\end{figure}


 В данном примере будет выбран пользователь \texttt{denilai}, который имеет привилегии супер-пользователя в данной базе данных и может вносить изменения в базу даннных, просматривать все записи в ней, создавать новые постоянные и временные таблицы, процедуры и макросы.



\subsubsection{Отображение записей таблиц}
После успешной авторизации пользователю будет предложено выбрать таблицу в базе данных, записи в которой будут отображены на экране. Поля таблиц могут содержать кириллические символы --- они будут корректно отображаться веб-приложением (см. Рисунки \ref{fig:select-1}, \ref{fig:select-log}). %\cite{converse2004php5} \cite{baldinLatex} \cite{d2004mysql} \cite{yurt2011dev}

% TODO: \usepackage{graphicx} required
\begin{figure}[h!]
	\centering
	\includegraphics[width=1\linewidth]{web-clent/select-1}
	\caption{Отображение записей таблицы Component}
	\label{fig:select-1}
\end{figure}

% TODO: \usepackage{graphicx} required
\begin{figure}[h!]
	\centering
	\includegraphics[width=1\linewidth]{web-clent/select-1}
	\caption{Отображение записей таблицы Log}
	\label{fig:select-log}
\end{figure}



\subsubsection{Добавление записей в в таблицы}
Ниже, под выведенными записями таблицы расположены поля для добавления значений в выбранную базу данных. В случае ввода корректных данных, после нажатия кнопки <<OK>>, поля будут немедленно добавлены в таблицу и отобразятся на экране (см. Рисунок \ref{fig:insert-1}).

% TODO: \usepackage{graphicx} required
\begin{figure}[h!]
	\centering
	\includegraphics[width=1\linewidth]{web-clent/insert-1}
	\caption{Добавление записей в таблицу Frame}
	\label{fig:insert-1}
\end{figure}


\subsubsection{Выполнение произвольных запросов}
В самом низу веб--страницы расположено текстовое поле для выполнения SQL--запросов. Запросы могут относиться не только к выбранной таблице или базе данных, а к любой таблице, права на взаимодействие с которой имеет выбранный пользователь. 

Текст, введенный пользователем, передается в виде строки на исполнение MySQL серверу. В случае ввода корректного запроса, он будет выполнен, о чем будет сообщено пользователю. 

С помощью данного поля также можно производить операции по удалению и вставке записей в таблицы, отображаемые на экране (см. Рисунок \ref{fig:request}).

% TODO: \usepackage{graphicx} required
\begin{figure}[h!]
	\centering
	\includegraphics[width=1\linewidth]{web-clent/request}
	\caption{Выполнение произвольных запросов}
	\label{fig:request}
\end{figure}

С помощью данного веб--клиента пользователь может просматривать записи, хранящиеся в базе данных, отмечать продвижение деталей по производственным цехам, отражать текущий статус сборки велосипеда, проверять наличие тех или иных компонентов на складе, получать полную информацию о них. Интерфейс отвечает требованиям, выдвинутым в задании: поддерживает русский язык, акцент в приложении сделан на содержании, дизайн интерфейса прост и лаконичен, чтобы упросить взаимодействие пользователя с системой.


%Воспользуемся декартовым произведением таблиц 







%\begin{lstlisting}
%create table posts (
%	id_post int (10) AUTO_INCREMENT,
%	message text NOT NULL,
%	id_author int (10) NOT NULL,
%	id_topic int (10) NOT NULL,
%	PRIMARY KEY (id_post),
%	FOREIGN KEY (id_author) REFERENCES users (id_user),
%	FOREIGN KEY (id_topic) REFERENCES topics (id_topic)
%);
%\end{lstlisting}

%\cite{d2004mysql},\cite{baldinLatex}, \cite{converse2004php5}, \cite{yurt2011dev}.

\backmatter %% Здесь заканчивается нумерованная часть документа и начинаются ссылки и
            %% заключение

\Conclusion % заключение к отчёту

В ходе выполнения данных практических работ были получены знания по написанию запросов на диалекте MySQL, отработаны навыки работы в средах ER Assistant и ERwin Data Modeler, навыки работы с интерфейсом командной строки MySQL CLI Client, а также с инструментом для визуального проектирования баз данных MySQL Workbench. 

Полученные знания были применены на практике для построения проектов баз данных, физические и логические модели баз данных. В последствии данные базы данных были реализованы в СУБД MySQL. Также было создано пользовательское веб~--~приложение, представляющее удобный интерфейс для взаимодействия с информационной системой и базой данных.

%%% Local Variables: 
%%% mode: latex
%%% TeX-master: "rpz"
%%% End: 


% % Список литературы при помощи BibTeX
% Юзать так:
%
% pdflatex rpz
% bibtex rpz
% pdflatex rpz

\bibliographystyle{gost780u}
\bibliography{course-plis/rpz}

%%% Local Variables: 
%%% mode: latex
%%% TeX-master: "rpz"
%%% End: 


\appendix   % Тут идут приложения
%
%\include{90-appendix1}
%\include{91-appendix2}

\end{document}

%%% Local Variables:
%%% mode: latex
%%% TeX-master: t
%%% End:
