% Также можно использовать \Referat, как в оригинале
\begin{abstract}
	
	
	Суть данной практической работы заключается в анализе и формировании требований к разрабатываемой информационной системе <<Электронный сборник лабораторных работ>>, в том числе требований к составу и содержанию работ по подготовке объекта автоматизации к вводу системы в действие, а также требований к документированию определении и формализации бизнес-ролей, 
	
	
	
	
	%Работа состоит из пояснительной записки и графической части. Графическая часть представляет собой описание реализуемого программного решения.  
	
	%Пояснительная записка содержит 3 раздела. 
	%В первом разделе освещается проблематика создания мультисинхронных цифровых устройств, то есть устройств, блоки которых используют более чем один синхросигнал, а также пути организации взаимной работы клоковых доменов.
	%Второй раздел посвящен проектированию разрабатываемого цифрового устройства по ресинхронизации данных. 
	%Третий раздел посвящен процессу реализации цифрового устройства на языке описания аппаратуры Verilog на основании спроектированных схем и моделей.
	

	
	Работу выполнил студент группы ИВБО-02-19 Денисов К.Ю. 
	
	%Руководитель --- Тарасов И.Е.
	
	 Данная практическая работа содержит \pageref*{LastPage}~страницы, \totfig~рисунков, \tottab~таблицы, 3 приложения. В работе использованы \totref~источников.
	 
	% Первое приложение содержит в себе исходные коды модулей на языке Verilog, разрабатываемых в ходе курсовой работы.
	 
	% Второе приложение содержит результаты симуляции цифрового устройства в САПР ISE Design Suite.
	 
	% Третье приложение представляет собой Руководство пользователя.
	
\end{abstract}

%%% Local Variables: 
%%% mode: latex
%%% TeX-master: "rpz"
%%% End: 
