\chapter{РЕАЛИЗАЦИЯ ЦИФРОВОГО УСТРОЙСТВА}

В данном разделе рассматривается процесс реализации устройства ресинхронизации данных и сопутствующих устройств средствами ISIM в САПР ISE Design Suite на языке описания аппаратуры Verilog.

\section{Реализация модуля устройства ресинхронизации}

Определим входные и выходные сигналы для модуля, реализующего устройство ресинхронизации данных (см. Таблицу \ref{tab:afifo-params}).

\begin{table}[htbp]
	\caption{Параметры файла afifo.v}
	\centering
	\fontsize{12}{16pt}
	\begin{tabular}{|c|c|c|}
		\hline
		\multicolumn{1}{|c}{\textbf{Hазвание}}& \multicolumn{1}{|c}{\textbf{Тип}} & \multicolumn{1}{|c|}{\textbf{Разрядность}} \\ \hline
		rden & input wire & 1 \\ \hline
		rrstn & input wire & 1 \\ \hline
		wclk & input wire & 1 \\ \hline
		wren & input wire & 1 \\ \hline
		wrstn & input wire & 1 \\ \hline
		rclk & input wire & 1 \\ \hline
		wdata & input wire &   [dw-1:0]  \\ \hline
		wfull & output	  reg  & 1 \\ \hline
		rempty & output	  reg	 & 1 \\ \hline
		rdready & output    wire  & 1 \\ \hline
		rdata; & output	  wire  & [(dw*4)-1:0]  \\ \hline
		wraddr & output    wire  & [aw-1:0]  \\ \hline
		rdaddr; & output    wire  & [aw-1:0]  \\ \hline
	\end{tabular}\label{tab:afifo-params}
\end{table}

Приведем содержание файла \textit{afifo.v}, в котором содержатся описания блоков чтения, записи, тактирования и адресации устройства ресинхронизации данных в Приложении \ref{cha:appendix1} на Рисунке \ref{lst:afifo}.


\section{Реализация модуля устройства ресинхронизации}
Для тестирования работы устройства опишем модуль, реализующий параметризованный делитель частоты. Приведем выходные и входные параметры данного модуля (см. Таблицу \ref{tab:freq-params}).

\begin{table}[htbp]
	\caption{Параметры файла freq\_div.v}
	\centering
	\fontsize{12}{16pt}\selectfont
	\begin{tabular}{|c|c|c|}
		\hline
		\multicolumn{1}{|c}{\textbf{Hазвание}}& \multicolumn{1}{|c}{\textbf{Тип}} & \multicolumn{1}{|c|}{\textbf{Разрядность}} \\ \hline
		rst & input wire & 1 \\ \hline
		clk & input wire & 1 \\ \hline
		co & output reg & 1 \\ \hline
	\end{tabular}
	\label{tab:freq-params}
\end{table}

Исходный код данного модуля приведен в Приложении \ref{cha:appendix1} на Рисунке \ref{lst:freq}.

\section{Тестирование модулей}
После реализации необходимых модулей, произведем тестирование корректности работы устройств, описав тестовый файл \textit{afifo\_test.v}. Исходный код приведен в Приложении \ref{cha:appendix1} на Рисунке \ref{lst:test}.

В качестве тестовых данных используем восьмиразрядные числа в шестнадцатеричном формате. Загрузку данных для удобства будем производить из файла data.txt. Чтение производится по 8 бит, тактирование по сигналу wclk. Запись производится в файл data\_out.txt по 32 бита, при этом становится наглядным то, что устройство позволяет производить ресинхронизацию данных, выполняя роль шины данных между разными клоковыми доменами. Содержание тестовых файлов приведено в Приложении \ref{cha:appendix1} на Рисунках \ref{lst:input} и \ref{lst:output}.

Произведем симуляцию реализованных модулей средствами ISim в САПР ISE Design Suite. Результаты симуляции приведены в Приложении \ref{cha:appendix2}.

При включении реализуемого устройства в состав более сложных систем, предварительно необходимо ознакомиться с Руководством пользователя, приведенным в Приложении \ref{cha:appendix3}, в котором описаны особенности реализации и использования данного аппаратного блока.









